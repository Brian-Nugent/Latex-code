\documentclass[a4paper]{article}

%% Language and font encodings
\usepackage[english]{babel}
\usepackage[utf8x]{inputenc}
\usepackage[T1]{fontenc}
\usepackage{yfonts}
\usepackage{frcursive}
\usepackage{aurical}
%\usepackage{LobsterTwo}
\usepackage{egothic}
\usepackage{miama}
\setlength\parindent{0pt}
\setlength\parskip{5pt}


%% Sets page size and margins
\usepackage[a4paper,top=3cm,bottom=2cm,left=2.5cm,right=2.5cm,marginparwidth=2cm]{geometry}

%% Useful packages
\usepackage{amsmath, latexsym, graphicx, amsthm, amssymb, scrextend, setspace,wasysym, marvosym, tikz}
\usepackage[colorinlistoftodos]{todonotes}
\usepackage[colorlinks=true, allcolors=blue]{hyperref}
\usepackage{tikz-cd}

\newcommand{\N}{\mathbb{N}}
\newcommand{\Z}{\mathbb{Z}}
\newcommand{\I}{\mathbb{I}}
\newcommand{\R}{\mathbb{R}}
\newcommand{\Q}{\mathbb{Q}}
\newcommand{\al}{\alpha}
\newcommand{\ao}{\alpha_0}
\newcommand{\an}{\alpha_n}
\newcommand{\zi}{\Z[i]}
\newcommand{\be}{\begin{equation}} 
\newcommand{\ee}{\end{equation}}   
\newcommand{\beq}{\begin{eqnarray*}} 
\newcommand{\eeq}{\end{eqnarray*}} 
\newcommand{\jump}{\vspace{0.3cm}}
\let\newproof\proof

\newtheorem{theorem}{Theorem}
\newtheorem{prop}[theorem]{Proposition}
\newtheorem{lemma}[theorem]{Lemma}
\newtheorem{definition}[theorem]{Definition}
\newtheorem{example}[theorem]{Example}
\newtheorem{counterexample}[theorem]{Counterexample}
\newtheorem{remark}[theorem]{Remark}
\newtheorem{corollary}[theorem]{Corollary}
\newtheorem{proposition}[theorem]{Proposition}
\newtheorem{conjecture}[theorem]{Conjecture}
\numberwithin{theorem}{section}


\begin{document}

\begin{center}
    {\huge Math 441: Introduction to Topology} 

    Brian Nugent
\end{center}
\jump

\tableofcontents

\section{Topology on the Real Numbers}

Before we define what a topology is formally, let's do a motivating example. You are hopefully familiar with continuous functions and limits from calculus. The usual definitions of continuous functions and limits use the fact that $\mathbb{R}$ has a notion of distance, given any two numbers $x$ and $y$ we can talk about their distance from each other $|x-y|$.

For example, the classic definition of a limit looks something like this: $x_1,x_2, \dots$ has limit $x$ if for any $\epsilon > 0$, there exists an $N \in \mathbb{N}$ such that $|x_n - x| < \epsilon$ for all $n \geq N$.

What may be surprising is that a notion of distance is not needed to talk about limits and continuous functions, all we need is a notion of "closeness". We achieve this by defining which subsets of $\mathbb{R}$ are \textit{open}.

Recall the definition of an open interval: $(a,b) = \{x \in \R : a < x < b \}$


\begin{definition}
A subset $U \subseteq \R$ is \textbf{open} if for all $x \in U$, there exists an $\epsilon > 0$ such that $(x-\epsilon,x+\epsilon) \subseteq U$.
\end{definition}

\textbf{Examples of open subsets of $\mathbb{R}$:} $\R, \varnothing, (0,1), (0, \infty), (0,1) \cup (2,3)$

We will now give new definitions of continuous functions and limits.

A map $f: \R \rightarrow \R$ is \textbf{continuous} if the preimage of any open subset is open. That is, $f^{-1}(U)$ is open for all open subsets $U \subseteq \R$.

A sequence $x_1,x_2,\ldots \in \mathbb{R}$ has limit $x$ if for any open $U \subseteq \mathbb{R}$ containing $x$, there exists an $N \in \mathbb{N}$ such that $x_i \in U$ for $i > N$.

We will show that these are equivalent to our original definition of continuity later, but first, let's prove some important properties of open subsets of $\mathbb{R}$.

\begin{prop}
Let $U$ and $V$ be open subsets of $\mathbb{R}$. Then $U \cup V$ and $U \cap V$ are both open.
\end{prop}

\begin{proof}
We prove $U \cup V$ is open first. Let $x \in U \cup V$. By definition of union, $x \in U$ or $x \in V$, without loss of generality we assume $x \in U$. Since $U$ is open, there exists an $\epsilon > 0$ such that $(x-\epsilon,x+\epsilon) \subseteq U$. Since $U \subseteq U \cup V$, $(x-\epsilon,x+\epsilon) \subseteq U \cup V$. Therefore $U \cup V$ is open.

Now we prove $U \cap V$ is open. Let $x \in U \cap V$. By definition of intersection, $x \in U$ and $x \in V$. So there exists $\epsilon_1,\epsilon_2 > 0$ such that $(x-\epsilon_1,x+\epsilon_1) \subseteq U$ and $(x-\epsilon_2,x+\epsilon_2) \subseteq V$. Now let $\epsilon = \textrm{ min} \{\epsilon_1,\epsilon_2\}$. Then $(x-\epsilon,x+\epsilon) \subseteq U$ and $(x-\epsilon,x+\epsilon) \subseteq V$, so $(x-\epsilon,x+\epsilon) \subseteq U \cap V$.

\end{proof}

Now notice that the proof that the union of two opens is open still works if we are unioning arbitrarily many open sets together. The proof that the intersection of two opens is open, however, required a step where we took the minimum of two real numbers. This proof will still work if we replace "two" with "finitely many", but it is not true that the intersection of arbitrarily many opens is open as we will see in the following example.

\textbf{Example:}
Let $U_n = (0,1+ \frac{1}{n})$ for $n \in \mathbb{N}$. For each $n \in \mathbb{N}$, $U_n$ is open, since it is an open interval. However,

$$ \bigcap U_n = (0,1] $$

which is not open because there is no neighborhood around 1 contained in $(0,1]$. Let's prove that $(0,1]$ is not open formally to get the hang of this style of proof. 

Suppose for the sake of contradiction that $(0,1]$ is open. Then since $1 \in (0,1]$, there exists an $\epsilon > 0$ such that $(1-\epsilon, 1+\epsilon) \subseteq (0,1]$. So $1+\frac{\epsilon}{2} \in (0,1]$. This is a contradiction since $1+\frac{\epsilon}{2} > 1$ and $(0,1] = \{x \in \mathbb{R}: 0<x\leq 1\}$.

This brings us to the definition of a topology.

\begin{definition} \label{top}
A \textbf{topology} on a set $X$ is a collection $\mathcal{T}$ of subsets of $X$ with the following properties:\\
(1) $\varnothing$ and $X$ are in $\mathcal{T}$.\\
(2) The union of elements of $\mathcal{T}$ is in $\mathcal{T}$.\\
(3) The intersection of finitely many elements of $\mathcal{T}$ is in $\mathcal{T}$.
\end{definition}
A set $X$ along with a topology $\mathcal{T}$ on $X$ is called a $\textbf{topological space}$.


We call the elements of $\mathcal{T}$ \textbf{open} subsets of $X$. So to define a topology on a set $X$ is to giving a description of which subsets of $X$ are open and showing that the properties of Definition \ref{top} hold.

The open subsets of $\mathbb{R}$ that we defined earlier form a topology called the standard topology on $\mathbb{R}$. Unless explicitly stated otherwise, when we refer to an open subset of $\mathbb{R}$, we always mean open in the standard topology.

Here is a useful lemma for constructing topologies:

\begin{lemma} \label{two}
Let $\mathcal{T}$ be a collection of subsets of $X$. Then $\mathcal{T}$ satisfies condition (3) if and only if the intersection of any two elements of $\mathcal{T}$ is in $\mathcal{T}$.
\end{lemma}

\begin{proof}
The forward direction is immediate so we assume that the intersection of any two elements of $\mathcal{T}$ is in $\mathcal{T}$ and prove (3) by induction on the number of elements $n$.

The base case is $n=1$ and there is nothing to prove.

We assume the intersection of any $n-1$ elements of $\mathcal{T}$ is in $\mathcal{T}$. Let $U_1, \ldots, U_n \in \mathcal{T}$. Then,

$$ U_1 \cap U_2 \cap \dots \cap U_n = U_1 \cap (U_2 \cap \dots \cap U_n) $$

$U_2 \cap \dots \cap U_n \in \mathcal{T}$ by induction and the intersection of any two subsets of $\mathcal{T}$ is in $\mathcal{T}$ by assumption, so $U_1 \cap U_2 \cap \dots \cap U_n \in \mathcal{T}$.

\end{proof}

\section{Topological Spaces and Bases}
Before we give examples, let's go over some lingo. If $(X,\mathcal{T})$ is a topological space, we say $U$ is an $\textbf{open}$ subset of $X$ if $U \in \mathcal{T}$. We call a subset $Z \subseteq X$ \textbf{closed} if $X - Z$ is open.

\textbf{Examples:} Given any set $X$, there are two somewhat trivial topologies that we can put on $X$. The collection $\mathcal{T} = \{\varnothing, X\}$ is a topology on $X$ called the \textbf{indiscrete topology}. The collection of all subsets of $X$ is called the \textbf{discrete topology}. 

Another odd topology we can put on any set $X$ is called the cofinite topology. This is where the open sets are those whose complements are finite (or all of $X$). In other words, it's the topology where the collection of finite subsets (along with $X$ itself) is the same as the collection of closed subsets. 



Often, the collection of all open sets of a topological space is too big to work with. Because of this, we define the notion of a \textit{basis} which is essentially a smaller collection of open sets that determines the rest of the topology.

\begin{definition} \label{basis}
Let $X$ be a set. A \textbf{basis} for a topology on $X$ is a collection $\mathcal{B}$ of subsets of X (called \textbf{basis elements}) such that:

(1) Every element of $X$ is contained in some $B \in \mathcal{B}$.

(2) If $x \in X$ is contained in two basis elements $B_1$ and $B_2$ then there exists a third basis element $B_3$ such that $x \in B_3$ and $B_3 \subseteq B_1 \cap B_2$.
\end{definition}

If $\mathcal{B}$ is a basis for a topology of $X$ then we define the \textbf{topology generated by $\mathcal{B}$}, denoted $\mathcal{T}(\mathcal{B})$, as follows:

We say a set $U \subseteq X$ is open if for all $x \in U$ there is a basis element $B \in \mathcal{B}$ such that $x \in B$ and $B \subseteq U$.

\begin{prop}
Let $\mathcal{B}$ be a basis for a topology on $X$. Then the topology $\mathcal{T}(\mathcal{B})$ generated by $\mathcal{B}$ is a topology on $X$.
\end{prop}

\begin{proof}
First we show that $\varnothing$ and $X$ are open. The fact that $X$ is open follows from (1) in the definition of a basis and $\varnothing$ is open trivially.

Let $\{U_i\}_{i \in I}$ be a collection of open subsets of $X$. Let $U = \underset{i \in I}{\bigcup} U_i$. Let $x \in U$. Then $x \in U_i$ for some $i \in I$. Since $U_i$ is open, there is a basis element $B$ containing $x$ such that $B \subseteq U_i$. Thus, $B \subseteq U$ proving that $U$ is open.

By Lemma 1.2, it is enough to show that the intersection of two opens is open. Let $U$ and $V$ be open subsets. Let $x \in U \cap V$. Since $U$ and $V$ are open, there exists basis elements $B_1,B_2 \in \mathcal{B}$ that contain $X$ such that $B_1 \in U$ and $B_2 \in V$. By (2) in the definition of a basis, there exists a basis element $B_3 \subseteq B_1 \cap B_2$ that contains $x$. Since $B_1 \cap B_2 \subseteq U \cap V$, $B_3 \subseteq U \cap V$. Thus $U \cap V$ is open.

\end{proof}

We will now prove two useful lemmas about bases.

\begin{lemma}
Let $\mathcal{B}$ be a basis for a topology on $X$. Then

$$ \mathcal{T}(\mathcal{B}) = \{\textrm{the collection of all unions of elements of } \mathcal{B}\}$$
\end{lemma}

\begin{proof}
By definition $\mathcal{T}(\mathcal{B})$ is the collection of sets $U$ with the property that for all $x \in U$, there is a $B \in \mathcal{B}$ containing $x$ such that $B \subseteq U$. So for each $x \in X$, choose a basis element $B_x$ containing $x$ that is contained in $U$. Then $\bigcup_{x \in X} = U$.

\end{proof}

Sometimes it is useful to find a basis for a topology that is already given. Suppose that $\mathcal{T}$ is a topology on $X$ and we find a subset $\mathcal{B} \subseteq \mathcal{T}$ such that $\mathcal{B}$ is a basis for a topology on $X$ and $\mathcal{T}(\mathcal{B}) = \mathcal{T}$. Then we say that $\mathcal{B}$ is a basis for $\mathcal{T}$. Our next lemma tells us how to identify these.

\begin{lemma}
Let $\mathcal{T}$ be a topology on $X$ and let $\mathcal{B} \subseteq \mathcal{T}$. Suppose the following condition holds:

\begin{center}
    For all $U \in \mathcal{T}$ and $x \in U$, there exists a $B \in \mathcal{B}$ such that $x \in B$ and $B \subseteq U$.
\end{center}

Then $\mathcal{B}$ is a basis for $\mathcal{T}$.

\end{lemma}

Try proving this one yourself! It is good to get used to unravelling all of the complicated definitions. Or you can read Lemma 13.2 in Munkres.

Using this lemma, try proving that the collection open intervals $(a,b)$ for all $a < b$ forms a basis for the standard topology on $\mathbb{R}$.

\section{Topology on $\mathbb{R}^n$}

We are now ready to define the most important topology, the standard topology on $\mathbb{R}^n$.

We recall the definition of the \textbf{norm} $|\cdot |$ on $\mathbb{R}^n$ by 

$$ |(x_1,\ldots,x_n)| = \sqrt{x_1^2+\dots+x_n^2} $$

The norm of a point is its distance from the origin. The \textbf{distance} between two points $x,y \in \mathbb{R}^n$ defined by $d(x,y) = |x-y|$

We recall without proof the most important properties of the function $d$,

\begin{prop} $\empty$ \\
(1) $d(x,y) \geq 0$ with equality if and only if $x=y$. \\
(2) $d(x,y) = d(y,x)$ for all $x,y \in \mathbb{R}^n$. \\
(3) $d(x,y) + d(y,z) \geq d(x,z)$, for all $x,y,z \in \mathbb{R}^n$. (Triangle Inequality) 
\end{prop}

A set along with a function $d: X \times X \rightarrow \mathbb{R}$ satisfying (1)-(3) is called a \textbf{metric space}. Many of the proofs we use for $\mathbb{R}^n$ work verbatim for metric spaces but we stick to $\mathbb{R}^n$ to keep the proofs easier to read.

Let $\epsilon > 0$ and let $p \in \mathbb{R}^n$, we define the \textbf{open ball of radius $\epsilon$ around p} by

$$ B_\epsilon(p) = \{x \in \mathbb{R}^n : d(p,x) < \epsilon\} $$

\begin{theorem}
The collection of subsets of $\mathbb{R}^n$ of the form $B_\epsilon(p)$ form a basis for a topology on $\mathbb{R}^n$.

\end{theorem}

\begin{proof}
First we show our collection satisfies (1) of Definition 2.1. Since $p \in B_1(p)$ for all $p \in \mathbb{R}^n$, (1) is satisfied. Now we show it satisfies (2) as well. Let $x \in B_\alpha(p) \cap B_\beta(q)$. Now let $\epsilon = \textrm{min}\{\alpha - d(p,x),\beta - d(q,x)\}$, we will show $B_\epsilon(x) \subseteq B_\alpha(p) \cap B_\beta(q)$ (drawing a picture will greatly help with understanding what is happening here). Let's show that $B_\epsilon(x) \subseteq B_\alpha(p)$. Let $y \in B_\epsilon(x)$. Then,

$$ d(p,y) \leq d(p,x) + d(x,y) < d(p,x) + (\alpha - d(p,x)) = \alpha $$

By the triangle inequality and the definition of $\epsilon$. Thus $y \in B_\alpha(p)$. The proof that $B_\epsilon(x) \subseteq B_\beta(q)$ is the same so have shown $B_\epsilon(x) \subseteq B_\alpha(p) \cap B_\beta(q)$.

\end{proof}

\textbf{Examples:} The following are open subsets of $\mathbb{R}^n$.

The upper open half space: $\mathbb{H} := \{(x,y) \in \mathbb{R}^2 : y > 0\}$

The open annulus: $\{x \in \mathbb{R}^n : 1 < |x| < 2\}$

The open triangle: $\{(x,y) \in \mathbb{R}^2 : x,y > 0, x+y<1\}$

\section{The Subspace Topology}

So far, we have seen relatively few examples of topological spaces, certainly not enough to convince you of the importance of the subject. Today we will see how to give a topology to any subset of a topological space. In other words, given a topology on a set $X$ and a subset $Z \subseteq X$, there is a natural way to "restrict" the topology to $Z$. This may seem minor at first, but considering that subsets of $\mathbb{R}^n$ are some of the most interesting in all of math (most objects in the world can be thought of as subsets of $\mathbb{R}^3$ for example), this is a pretty big step.

\begin{definition}
Let $X$ be a topological space with topology $\mathcal{T}$ and let $Z$ be a subset of $X$. The collection

$$ \mathcal{T}_{Z} = \{U \cap Z : U \in \mathcal{T}\} $$

is called the \textbf{subspace topology} on $Z$.

\end{definition}

\begin{theorem}
The subspace topology is indeed a topology.
\end{theorem}

\begin{proof}
The fact that $Z,\varnothing \in \mathcal{T}_{Z}$ follows from the fact that $X,\varnothing \in \mathcal{T}$.

Let $\{U_i\}$ be a collection of elements of $\mathcal{T}_{Z}$. Then there exist $V_i \in \mathcal{T}$ such that $U_i = V_i \cap Z$. We see that

$$ \underset{i}{\bigcup} U = \underset{i}{\bigcup} (V_i \cap Z) = \left(\underset{i}{\bigcup} V_i \right) \cap Z$$

So $\underset{i}{\bigcup} U \in \mathcal{T}_{Z}$.

Let $U_1,U_2 \in \mathcal{T}_{Z}$. Then there exists $V_1,V_2 \in \mathcal{T}$ such that $U_i = V_i \cap Z$. We see that

$$ U_1 \cap U_2 = (V_1 \cap Z) \cap (V_2 \cap Z) = (V_1 \cap V_2) \cap Z $$

So $U_1 \cap U_2 \in \mathcal{T}_{Z}$.

\end{proof}

If $Z$ is a subset of a topological space $X$ and we say $U$ is an open subset of $Z$, we always mean in the subspace topology.

\textbf{Example:} Let us look at $[0,1]$ as a subspace of $\mathbb{R}$ and determine which subsets are open in the subspace topology. We see that $(0.5,1] = (0.5,2) \cap [0,1]$, so it is open even though it is not open in $\mathbb{R}$. Similarly $[0,a)$ and $(a,1]$ are open for all $a \in (0,1)$. We also see that if $0<a<b<1$, then $(a,b) = (a,b) \cap [0,1]$, so it remains open in $[0,1]$. These subsets and unions of them are all of the open subsets of $[0,1]$ as we will see in the following lemma. \\

\begin{lemma}
Let $X$ be a topological space with a basis $\mathcal{B}$ and let $Z$ be a subset of $X$. Define,

$$ \mathcal{B}_{Z} = \{B \cap Z : B \in \mathcal{B}\} $$

Then $\mathcal{B}_{Z}$ is a basis for the subspace topology on $Z$.

\end{lemma}

\begin{proof}
We prove this by showing the condition of lemma 2.4 holds. Let $U \in \mathcal{T}_Z$ and let $x \in U$. By definition, $U = V \cap Z$ where $V \in \mathcal{T}$. Since $\mathcal{B}$, is a basis, there exists a $B \in \mathcal{B}$ such that $x \in B$ and $B \subseteq V$. Then $B \cap Z \in \mathcal{B}_Z$ and we have proven the condition of lemma 2.4.
\end{proof}

\textbf{Exercise:} Show that the topology from Homework problem 1-1 was the subspace topology on $U$. In particular, if $U$ is an open subset of a topological space $X$, then every open subset of $U$ is an open subset of $X$.

\section{The Product Topology}
Let $X$ and $Y$ be sets. We recall the definition of the cartesian product $X \times Y$,

$$ X \times Y = \{(x,y):x \in X , y \in Y\} $$

If $X$ and $Y$ are topological spaces, we would like to give a topology to their product. We do so by providing a basis. Let $\mathcal{B}$ be the collection of subsets of $X \times Y$ of the form $U \times V$ where $U$ is an open subset of $X$ and $Y$ is an open subset of $Y$.

\begin{theorem}
$\mathcal{B}$ is a basis for a topology on $X \times Y$.
\end{theorem}

\begin{proof}
Part (1) of showing $\mathcal{B}$ is a basis follows from the fact that $X \times Y$ is a basis element. Part (2) follows from the fact that, in this case, the intersection of two basis elements actually is another basis element as we see below,

$$ (U_1 \times V_1) \cap (U_2 \times V_2) = (U_1 \cap U_2) \times (V_1 \cap V_2) $$
\end{proof}

The topology generated by this basis is called the product topology on $X \times Y$.

If $X$ and $Y$ have bases, then we can get an even simpler basis for $X \times Y$.

\begin{theorem}
Let $X$ be a topological space with basis $\mathcal{B}$ and let $Y$ be a topological space with basis $\mathcal{C}$. Then the collection

$$\mathcal{D} = \{B \times C: B \in \mathcal{B}, C \in \mathcal{C}\} $$

is a basis for the product topology on $X \times Y$.

\end{theorem}

\begin{proof}
We will use Lemma 2.4 to prove this. Let $A$ be an open subset of $X \times Y$ and let $(x,y) \in A$. By definition of the product topology, $(x,y) \in U \times V \subseteq A$ for some open subsets $U$ of $X$ and $V$ of $Y$. Thus, there is a basis element $B \in \mathcal{B}$ such that $x \in B \subseteq U$. Similarly, there is a basis element $C \in \mathcal{C}$ such that $y \in C \subseteq V$. Thus, $(x,y) \in B \times C \subseteq U \times V$.
\end{proof}


\begin{theorem}
The product topology on $\mathbb{R}^2$ is the same as the standard topology.
\end{theorem}

\begin{proof}
This proof will not be as rigorous as most of the proofs we do, its purpose is so that if you are interested in the proof, you can use this as a guide to proving it yourself (with lots of pictures!)

Let $\mathcal{B}_1$ be the collection of open balls in $\mathbb{R}^2$, which is a basis for the standard topology. Let $\mathcal{B}_2$ be the collection of sets of the form $(a,b)\times (c,d)$, which is a basis for the product topology on $\mathbb{R}^2$ by Theorem 5.2.

To show that the topologies are the same, it is enough to show that the elements of $\mathcal{B}_1$ are open in the product topology and the elements of $\mathcal{B}_2$ are open in the standard topology (think about why this is true). 

Let $x \in B_\epsilon(p)$. Then $B_\delta(x) \subseteq B_\epsilon(p)$ for some $\delta > 0$. Let $x = (a,b)$. We see that $(a-\frac{\delta}{\sqrt{2}},a+\frac{\delta}{\sqrt{2}}) \times (b-\frac{\delta}{\sqrt{2}},b+\frac{\delta}{\sqrt{2}}) \subseteq B_\delta(x)$, so $B_\epsilon(p)$ is open in the product topology by Homework 1-3 (we showed every point is contained in an open set).

We do the same trick with $(a,b)\times (c,d)$. Let $(x,y) \in (a,b)\times (c,d)$. Let $\epsilon = \textrm{min}\{x-a,b-x,y-c,d-y  \}$. Then we see that $B_\epsilon(x) \subseteq (a,b)\times (c,d)$.

\end{proof}

\section{Closed Sets and Limit Points}

We will take a quick detour from defining different topologies to talk about the closed sets and limit points. We recall that a subset $Z$ of a topological space $X$ is closed if it's complement is open, that is, if $X - Z$ is open. 

\textbf{Example:} Let $X = [0,1] \cup (2,3)$. We see that the subset $[0,1]$ is closed since it's complement $(2,3)$ is open. We see that $[0,1]$ is open as well, since $[0,1] = (-\frac{1}{2}, \frac{3}{2})$. Subsets that are open and closed are sometimes called \textbf{clopen}.

Since closed sets are just the complements of the open sets, pretty much every statement we prove about open sets has a complementary version about closed sets. In fact, we could have defined a topology in terms of closed set instead of open sets, where the two conditions are swapped, as we see below:

\begin{prop}
Let $X$ be a topological space. The following hold:
\begin{enumerate}
    \item $X$ and $\varnothing$ are closed 
    \item Any intersection of closed subsets of $X$ is closed 
    \item Any finite union of closed subsets of $X$ is closed
\end{enumerate}
Defining a topology is equivalent to declaring which subsets are closed and showing these three statements hold.

\end{prop}

Unsurprisingly, the same rule that applies to open sets for the subspace topology applies to closed sets:

\begin{lemma}
Let $Z$ be a subspace of a topological space $X$. A subset $Y \subseteq Z$ is closed in $Z$ if and only if there is a closed subset of $Y'$ of $X$ such that $Y = Y' \cap Z$.
\end{lemma}

\begin{proof}
Let $Y$ be closed in $Z$. 

$\Leftrightarrow$ $Z-Y$ is open in $Y$

$\Leftrightarrow$ $Z-Y = U \cap Z$ for some $U$ open in $X$

$\Leftrightarrow$ $Y = (X-U) \cap Z$

\end{proof}

Let $A$ be a subset of a topological space $X$. In Homework problem 1-4, we defined the \textbf{interior} of $A$, denoted $A^o$, to be the union of all open subsets of $X$ that are contained in $A$ (note that we said open subsets of $X$ not $A$). This is essentially the "largest" open subset of $X$ that is contained in $A$.

\textbf{Example:} If $X = \mathbb{R}$ and $A = [0,1]$, then the interior of $A$ is $(0,1)$. Clearly $(0,1) \subseteq A^o$ since $(0,1)$ is open in $\mathbb{R}$ and contained in $[0,1]$, so we only need to argue that 0 and 1 are not in $A^o$. Suppose for the sake of contradiction that $0 \in A^o$. Then since $A^o$ is open, there exists an $\epsilon > 0$ such that $(-\epsilon,\epsilon) \subseteq A^o$. This implies $-\frac{\epsilon}{2}$ is contained in $A$, which is a contradiction.

We now define the complementary version of the interior. The \textbf{closure} of $A$, denoted $\overline{A}$, is the intersection of all closed subsets of $X$ that contain $A$. It is the "smallest" closed subset of $X$ that contains $A$ which is made formal by the following theorem:

\begin{theorem} \label{closure}
Let $A$ be a subset of a topological space $X$. The following are true:
\begin{enumerate}
    \item $A \subseteq \overline{A}$
    \item $\overline{A}$ is a closed subset of $X$
    \item If $Z$ is a closed subset of $X$ and $A \subseteq Z$, then $\overline{A} \subseteq Z$
\end{enumerate}
\end{theorem}


We obtain the following useful corollary:
\begin{corollary}
Let $A$ be a subset of a topological space $X$. Then,
\begin{enumerate}
    \item $A$ is closed in $X$ if and only if $A = \overline{A}$
    \item $A$ is open in $X$ if and only if $A = A^o$
\end{enumerate}
\end{corollary}
\begin{proof}
Lets prove 1. If $A = \overline{A}$, then $A$ is closed by Theorem \ref{closure} part 2. 

Now assume $A$ is closed. Then $\overline{A} \subseteq A$ by Theorem \ref{closure} part 3 and $A \subseteq \overline{A}$ by Theorem \ref{closure} part 1. So $A = \overline{A}$. The proof of part 2 is similar.

\end{proof}

We say a set $A$ \textbf{intersects} a set $B$ if $A \cap B$ is nonempty. If $X$ is a topological space and $x \in X$ a point, a \textbf{neighborhood} of $x$ is any open subset of $X$ containing $x$.

\begin{lemma} \label{lem}
Let $A$ be a subset of a topological space $X$. Then,
\begin{enumerate}
    \item $x \in \overline{A}$ if and only if $A$ intersects every neighborhood of $x$.
    \item If the topology is given by a basis then $x \in \overline{A}$ if and only if every basis element containing $x$ intersects $A$.
\end{enumerate}
\end{lemma}

\begin{proof}
We will prove (1) using the contrapostive, the proof of (2) is similar. So we must show that $x$ has a neighborhood that does not intersect $A$ if and only if $x \not \in \overline{A}$.

($\Leftarrow$) Assume $x \not \in \overline{A}$. Then $U = X - \overline{A}$ is a neighborhood of $x$ that does not intersect $A$.

($\Rightarrow$) Let $U$ be a neighborhood of $x$ that does not intersect $A$. Then $X - U$ is a closed set that contains $A$, so $\overline{A} \subseteq X-U$. Since $x \not \in X -U$, $x \not \in \overline{A}$.

\end{proof}

\begin{definition}
Let $A$ be a subset of a topological space $X$. We say a point $x \in X$ is a limit point of $A$ if every neighborhood of $x$ intersects $A$ at some point \textit{other than} $x$.

\end{definition}

\textbf{Example:} The point 1 is a limit point of $(0,1)$ but 1 is not a limit point of $\{1\}$. This shows that limit points of $A$ are not always contained in $A$ and that elements of $A$ are not always limit points.

\begin{theorem} \label{limpts}
Let $A$ be a subset of a topological space $X$ and let $A'$ be the set of limit points of $A$. Then

$$ \overline{A} = A \cup A' $$
\end{theorem}

\begin{proof}
We see that $A \subseteq \overline{A}$ by Theorem \ref{closure} and $A' \subseteq \overline{A}$ by Lemma \ref{lem}, so $A \cup A' \subseteq \overline{A}$.

Now let $a \in \overline{A}$. If $a \in A$ then $a \in A \cup A'$ so assume $a \not \in A$. By Lemma \ref{lem}, every neighborhood of $a$ intersects $A$ and since $a \not \in A$ it must intersect at a point other than $a$. Therefore $a \in A'$.

\end{proof}

\begin{corollary}
Let $A$ be a subset of a topological space. $A$ is closed if and only if it contains all of its limit points.
\end{corollary}

\begin{proof}
$A$ is closed if and only if $A = \overline{A}$ which happens if and only if $A$ contains all of its limit points by Theorem \ref{limpts}.
\end{proof}

\section{Hausdorff Spaces}

We now come to our first "niceness" condition for topological spaces. Many of the things that are true for $\mathbb{R}^n$ that feel intuitive simply do not hold if we don't put any restrictions on our spaces. For example, in $\mathbb{R}^n$ the set containing any single point is closed and sequences converge to at most one limit. Looking at any set with the indiscrete topology shows that this is not true in general.

We say two sets $A$ and $B$ are \textbf{disjoint} if $A \cap B = \varnothing$.

\begin{definition}
A topological space $X$ is \textbf{Hausdorff} if for all $x,y \in X$, there exists disjoint open sets $U$ and $V$ such that $x \in U$ and $y \in V$.
\end{definition}

\begin{theorem}
Let $X$ be a Hausdorff space and let $x \in X$. Then $\{x\}$ is closed.
\end{theorem}

\begin{proof}
Let $y$ be an element of $X-\{x\}$. Since $X$ is Hausdorff, there exist disjoint open sets $U,V$ of $X$ such that $y \in U$ and $x \in V$. In particular, $U \subseteq X - \{x\}$. By Homework problem 1-3, $X-\{x\}$ is open, so $\{x\}$ is closed.
\end{proof}

\begin{definition}
Let $X$ be a topological space. A sequence of points $x_1,x_2,\dots$ converges to $x$ if for every neighborhood $U$ of $x$, there exists a $N \in \mathbb{N}$ such that $x_n \in U$ for all $n \in \mathbb{N}$. 
\end{definition}

\begin{theorem}
Let $X$ be a Hausdorff space. A sequence of points of $X$ converges to at most one point of $X$.
\end{theorem}

\begin{proof}
Let $x_1,x_2,\dots$ be a sequence of points of $X$ and assume it converges to $x$. Let $y \neq x$. Since $X$ is Hausdorff, there exists disjoint neighborhoods $U$ and $V$ containing $x$ and $y$ respectively. By the definition of convergence, $U$ contains all but finitely many of the $x_i$, so $V$ contains at most finitely many of the $x_i$. Therefore the sequence cannot converge to $y$.
\end{proof}

\textbf{Exercise:} The product of two Hausdorff spaces is Hausdorff and a subspace of a Hausdorff space is Hausdorff.

\textbf{Examples:} Most of the topological spaces we care about are Hausdorff, for example, $\mathbb{R}^n$ and all of it's subspaces.

An example of a topological space that is not Hausdorff is an infinite set $X$ with the cofinite topology, we can see that any two nonempty subsets will intersect.

\section{Continuous Functions}

Continuous functions are fundamental to many areas of mathematics. In fact, the reason we defined topological spaces the way we did was to make them the most general objects for which continuous functions make sense.

\begin{definition}
Let $X$ and $Y$ be topological spaces. A function $f: X \rightarrow Y$ is \textbf{continuous} if for every open subset of $Y$, $U$, the preimage $f^{-1}(U)$ is an open subset of $X$.
\end{definition}

Note: We sometimes call functions maps instead, there is no difference between the two.

Like most things in topology, it suffices to check that a function is continuous on a basis:

\begin{proposition}
Let $X$ and $Y$ be topological spaces and let $\mathcal{Y}$ be a basis for $Y$. Then $f$ is continuous if and only if $f^{-1}(B)$ is open for all $B \in \mathcal{B}$.
\end{proposition}

\begin{proof}
This follows from the fact that
$$f^{-1}\left( \bigcup_{i \in I} B_i  \right) =  \bigcup_{i \in I} f^{-1}(B_i) $$
\end{proof}

\textbf{Note:} In calculus, you have seen many examples of continuous functions from $\mathbb{R}^n$ to $\mathbb{R}$. The definition of continuity used in calculus is all the "$\epsilon-\delta$ definition" and it is equivalent to our definition when everything has the standard topology. For this reason, we do not need to prove most functions from $\mathbb{R}^n$ to $\mathbb{R}$ are continuous, we can just cite calculus.
 
\textbf{Examples:} $f(x) = \sin (x)$

$f(x) = e^x$

$f(x,y) = ye^{x^2} + \frac{\sin (x)}{y^2 + 1}$

\begin{definition}
A bijective continuous function $f: X \rightarrow Y$ is called a \textbf{homeomorphism} if its inverse $f^{-1}: Y \rightarrow X$ is continuous as well. If $f: X \rightarrow Y$ is a homeomorphism, we say $X$ and $Y$ are homeomorphic.
\end{definition}

Let's examine a homeomorphism $f: X \rightarrow Y$ more closely. Let $U$ be an open subset of $X$. Saying that $f^{-1}$ is continuous means that the preimage of $U$ under this map is open. But the preimage of $U$ under $f^{-1}$ is $f(U)$. So we see that $f$ being a homeomorphism is equivalent to $f$ being a bijection with the property: $U$ is open if and only if $f(U)$ is open.

Because of this, we think of homeomorphic spaces as being "the same" at least as far as topology is concerned.

\begin{corollary}
A composition of homeomorphisms is a homeomorphism.
\end{corollary}
\begin{proof}
Let $f: X \rightarrow Y$ and $g: Y \rightarrow Z$ be homeomorphisms. Then $U$ is open in $X$ if and only if $f(U)$ is open in $Y$ if and only if $g(f(U))$ is open in $Z$. Thus $g \circ f$ is a homeomorphism.

\end{proof}

Let $X$ be a topological space. A property of $X$ is called a \textbf{topological property} if any space homeomorphic to $X$ has the property.

\begin{theorem}
Being Hausdorff is a topological property.
\end{theorem}

\begin{proof}
Let $X$ be a Hausdorff space and let $f: X \rightarrow Y$ be a homeomorphism. We must show that $Y$ is Hausdorff. Let $x,y \in Y$. Pick disjoint neighborhoods $U$ and $V$ for $f^{-1}(x)$ and $f^{-1}(y)$. Then $f(U)$ and $f(V)$ are disjoint neighborhoods for $x$ and $y$.
\end{proof}

\textbf{Example:} An example of something that is not a topological property is "boundedness". For example, $(0,1)$ is bounded and $\mathbb{R}$ is not. In homework 3, you will show that $(0,1)$ is homeomorphic to $\mathbb{R}$ which shows boundedness cannot be a topological property.

If $f: X \rightarrow Y$ is an injective continuous map and the restriction of the codomain of $f$ to it's image with the subspace topology is a homeomorphism then $f$ is called a \textbf{topological embedding} (or just an \textbf{embedding}).

\textbf{Example:} Define $f: \mathbb{R} \rightarrow \mathbb{R}$ by $f(x) = 2x$. Now define $g(x) = \frac{x}{2}$. Then $f(g(x)) = f \left( \frac{x}{2} \right) = x$ and $g(f(x)) = g(2x) = x$, so $g$ is the inverse of $f$. Both $f$ and $g$ are continuous by calculus, so they are homeomorphisms. Note that the fact that they are bijective follows from the fact that they have an inverse.

\textbf{Example:} For an example of a bijective continuous function that is not a homeomorphism we can look at the identity function from $\mathbb{R}$ to $\mathbb{R}_{\textrm{cof}}$ where $\mathbb{R}_{\textrm{cof}}$ is $\mathbb{R}$ with the cofinite topology.

For a more interesting example, let's define a map $f: [0,1) \rightarrow \mathbb{S}^1$ by 

$$ f(x) = (\cos (2\pi x),\sin (2\pi x)) $$

Or in complex notation, this looks like $f(x) = e^{2\pi x}$.

This function is continuous by calculus and you can check that it is bijective. We see that the image of the open set $[0,\frac{1}{2})$ is not open in $\mathbb{S}^1$ so $f$ is not a homeomorphism.

Note that just because $f$ is not a homeomorphism does not mean that $[0,1)$ and $\mathbb{S}^1$ are not homeomorphic. In general, showing two spaces are not homeomorphic is a hard problem.

\textbf{Hard Problem:} Which topological spaces that we've talked about so far are homeomorphic and which are not? Try proving that $\mathbb{R}^2$ and $\mathbb{R}^3$ are not homeomorphic and you'll realize it is much harder than it seems. One of the main reason for the machinery we are going to learn in this course is to be able to differentiate different topological spaces.

\textbf{Example:} A subspace of $\mathbb{R}^3$ homeomorphic to $\mathbb{S}^1$ is called a \textbf{knot}. We say two knots $K_1$ and $K_2$ are equivalent if there exists a homeomorphism $f: \mathbb{R}^3 \rightarrow \mathbb{R}^3$ such that $f(K_1) = K_2$. Determining which knots are equivalent is the main topic of Knot Theory. Later in this class, we will prove that any injective continuous function $f: \mathbb{S}^1 \rightarrow \mathbb{R}^3$ is a topological embedding and therefore the image of $f$ is a knot.

\jump

\begin{theorem}
A function $f: X \rightarrow Y$ is continuous if and only if $f^{-1}(A)$ is closed for all closed subsets $A$ of $Y$.
\end{theorem}

\begin{proof}
This follows from the fact that 

$$ f^{-1}(Y-A) = X - f^{-1}(A) $$
\end{proof}

\jump

Rules for constructing continuous functions: Let $X,Y$ and $Z$ be topological spaces.  

\jump

(a) (Constant function) If $f: X \rightarrow Y$ maps all of $X$ to a single point $y_0$ of $Y$, then $f$ is continuous. 

(b) (Inclusion) If $A$ is a subspace of $X$, the inclution function $i: A \rightarrow X$ is continuous.

(c) (Composition) If $f: X \rightarrow Y$ and $g: Y \rightarrow Z$ are continuous, then $g \circ f: X \rightarrow Z$ is continuous.

(d) (Restricting the Domain) If $f: X \rightarrow Y$ is continuous and $A$ is a subspace of $X$, then $f_{|A}: A \rightarrow Y$ is continuous.

(e) (Restricting the Codomain) If $f: X \rightarrow Y$ is continuous and $Z$ is a subspace of $Y$ containing $f(X)$, then the restricted function $f: X \rightarrow Z$ is continuous.

(f) (Expanding the Codomain) If $f: X \rightarrow Y$ is continuous and $Y$ is a subspace of $Z$, then $f: X \rightarrow Z$ is continuous.

(g) (Continuity is Local) Let $f: X \rightarrow Y$ be a function. If $X$ can be written as a union of open sets $X = \bigcup U_\alpha$ and $f_{U_\alpha}: U_\alpha \rightarrow Y$ is continuous for each $\alpha$, then $f$ is continuous.

(h) (Projection) Let $\pi_1: X \times Y \rightarrow X$ be the function defined by $\pi_1(x,y) = x$. Then $\pi_1$ is continuous.

We will do proofs of many of these in class. To see written proofs, look at page 108 of Munkres.

\jump

\begin{theorem}[\textbf{Gluing Lemma}]
Let $X = A \cup B$ where $A$ and $B$ are closed subsets. Let $f:A \rightarrow Y$ and $g: B \rightarrow Y$ be continuous and assume $f(x) = g(x)$ for all $x \in A \cap B$. Then there exists a continuous function $h: X \rightarrow Y$ obtained by setting $h(x) = f(x)$ for $x \in A$ and $h(x) = g(x)$ for $x \in B$.
\end{theorem}

\begin{proof}
By theorem 8.6 it suffices to show that the preimage of a closed set is closed. Let $K \in Y$ be closed. Then

$$ h^{-1}(K) = f^{-1}(K) \cup g^{-1}(K) $$

Since a finite union of closed sets is closed, we are done.
\end{proof}

Note: The same theorem holds if $A$ and $B$ are replaced by open sets by (g) in the rules for constructing continuous functions.

The following is called the universal property of the product. 

\begin{theorem}[\textbf{Universal Property of the Product Topology}]
Let $A,X,Y$ be topological spaces. Let $f: A \rightarrow X \times Y$ be a function given by,

$$ f(a) = (f_1(a),f_2(a)) $$

where $f_1 : A \rightarrow X$ and $f_2 : A \rightarrow Y$.

Then $f$ is continuous if and only if $f_1$ and $f_2$ are.
 
\end{theorem}

\begin{proof}
($\Rightarrow$) We assume $f$ is continuous. We see that $f_1 = \pi_1 \circ f$ where $\pi_1$ is the projection from (h) above. So $f_1$ is continuous by (c).

($\Leftarrow$) By Proposition 8.2, it suffices to check continuity on sets of the form $U \times V$ where $U$ is open in $X$ and $V$ is open in $Y$. We see that 

$$ f^{-1}(U \times Y) = f_1^{-1}(U) $$

which is open by continuity of $f_1$. Similarly $f^{-1}(X \times V)$ is open. Since 

$$ f^{-1}(U \times V) = f^{-1}(U \times Y) \cap f^{-1}(X \cap V) $$

$f^{-1}(U \times V)$ is open.

\end{proof}

\section{Infinite Products}

For the sake of completeness, we mention the product topology for infinite products. There will be no homework or test questions on it. We will restrict our attention to countable products to make the notation simpler but note that everything done in this section works the same way for arbitrary products.

We recall that the sets of the form $U \times V$ where $U$ is open in $X$ and $V$ is open in $Y$ formed a basis for a topology on $X \times Y$ which we called the product topology. 

Now given topological spaces $X_1,X_2 \dots$, we could try to do the same thing and define a collection of subsets of $X_1 \times X_2 \times \dots$ as the subsets of the form $U_1 \times U_2 \times \dots$ where $U_i$ is open in $X_i$. This will indeed give us the basis for a topology, we call this the \textbf{box topology}. We will see why this is not a good topology on the product in a moment.

Now let us think about the construction of the product topology on $X \times Y$ in a different way. The product comes equipped with two projections $\pi_1: X \times Y \rightarrow X$ and $\pi_2 \rightarrow Y$ which we would like to be continuous. For that to be true, for each $U$ open in $X$ and $V$ open in $Y$ we must make $\pi_1^{-1}(U) = U \times Y$ and $\pi_2^{-1}(V) X \times V$ open in $X \times Y$. If those are both open then their intersection must be open as well,

$$ (U \times Y) \cap (X \times V) = U \times V $$

So we take the topology on $X \times Y$ to be the one generated by subsets of this form and we get a really nice universal property since the open sets essentially defined to make the projections continuous.

To get the product topology for an infinite product, we replicate this construction. So we see that sets of the form 

$$ \pi_1^{-1}(U) = U_1 \times X_2 \times X_3 \times \dots $$

should be open in the product and similarly for the other projections. If these are open, then \textbf{FINITE} intersections of them should be open as well. A finite intersection of sets of this form will look like,

$$ U_1 \times U_2 \times \dots $$

where $U_i = X_i$ for all but finitely many $i$. One can show this forms the basis for a topology, which we call the \textbf{product topology}.

\begin{theorem}[Universal Property of the Product Topology]
Let $f: A \rightarrow X_1 \times X_2 \times \dots$ be a function given by 

$$ f(a) = (f_1(a),f_2(a),\dots) $$

Then $f$ is continuous if and only if $f_i$ is continuous for all $i$.

\end{theorem}


\section{The Quotient Topology}

The idea behind the quotient topology is different than most of the topologies we've seen so far. For the quotient topology, we would like to take a topological space and "glue" some of it's points together to obtain a new space. Before we talk about gluing topological spaces, lets talk about how to glue points in a set together.

\begin{definition}
An \textbf{equivalence relation} is a relation $\sim$ on a set $X$ that satisfies the following conditions for all $a,b,c\in X$:
\begin{enumerate}
    \item $a \sim a$
    \item $a \sim b \Leftrightarrow b \sim a$
    \item $a \sim b$ and $b \sim c \Rightarrow a \sim c$
\end{enumerate}
\end{definition}

We will often define an equivalence relation by only declaring some of the relations we want to hold. For example, if we want the equivalence relation that glues the two endpoints of $[0,1]$ together, we will just say "let $\sim$ be the equivalence relation where $0 \sim 1$". It is implied that $1 \sim 0$ and $a \sim a$ for all $a \in [0,1]$.

Given an equivalence relation $\sim$ on $X$ and an element $x \in X$, the \textbf{equivalence class} of $x$ is the set of all points $y \in X$ such that $x \sim y$ holds. We denote it by $[x]$.

$$ [x] = \{y \in X : x \sim y\} $$

Given an equivalence relation $\sim$ on $X$, we can look at the set $X^*$ of equivalence classes of $\sim$. Then we obtain a natural surjective function $q: X \rightarrow X^*$ defined by $q(x) = [x]$.

Now we will show that every surjective function of sets is "essentially" of this form. Let $q: X \rightarrow Y$ be a surjective function. Then we can define an equivalence relation on $X$ by declaring that $x \sim y$ if and only if $q(x) = q(y)$. Then we see that there is a natural bijection $f: X^* \rightarrow Y$ defined by $f([x]) = q(x)$.

\begin{center}
\begin{tikzcd}
X \arrow[d, two heads] \arrow[rd, two heads] &     \\
Y \arrow[r, leftrightarrow]                                  & X^*
\end{tikzcd}
\end{center}

What is happening here is we are "gluing" $X$ according to $\sim$. If $x\sim y$, then $x$ and $y$ become the same point in $X^*$. Now we learn how to glue topological spaces. First we define what we would like to happen.

\begin{definition}
Let $q: X \rightarrow Y$ be a surjective continuous map. We say $q$ is a \textbf{quotient map} if the following condition holds for every subset $U$ of $Y$:
\begin{center}
$U$ is open if and only if $f^{-1}(U)$ is open
\end{center}
\end{definition}

Now we see that one direction of the condition always holds by the definition of continuity. The important part is that we are saying that if $q^{-1}(U)$ is open then $U$ is necessarily open. In other words, to check if a subset of $Y$ is open, we can pull it back to $X$ and check there. So the topology on $Y$ is completely determined by the topology on $X$. Now we show that there is a natural way to create quotient maps.

\begin{theorem}
Let $X$ be a topological space and $q: X \rightarrow A$ a surjective function onto a set $A$. Then there exists a unique topology on $A$ that makes $q$ a quotient map. We call this topology the \textbf{quotient topology}.
\end{theorem}

\begin{proof}
We define $\mathcal{T}$ to be the collection of subsets of $A$, $U$, with the property that $q^{-1}(U)$ is open in $X$. Then you can check that $\mathcal{T}$ is indeed a topology on $A$ and the discussion above shows that it is unique.
\end{proof}

Now we can glue topological spaces pretty easily. We just take a topological space $X$ and define our equivalence relation $\sim$ by saying that $x \sim y$ if we want to glue $x$ and $y$. Then we give $X^*$ the quotient topology using our natural surjection $q: X \rightarrow X^*$.

\textbf{Example:} Let $X = [0,1]$ and $\sim$ be the equivalence relation where $0 \sim 1$. Then the quotient topology on $X^*$ is homeomorphic to $\mathbb{S}^1$ (we will show this later).

\textbf{Example:} Let $S = [0,1] \times [0,1]$. We would like to glue the sides of the square $S$ together to get a donut. So let's define an equivalence relation on $S$ by declaring that $(0,a) \sim (1,a)$ and $(a,0) \sim (a,1)$ for all $a \in [0,1]$. Then we can look at the quotient space $S^*$ obtained by gluing $S$ according to our relation $\sim$.

\textbf{Example:} Now what if we did the same construction as above but glued one of the sides in backwards? Once again, let $S = [0,1] \times [0,1]$. We define $\sim$ by declaring that $(0,a) \sim (1,-a)$ and $(a,0) \sim (a,1)$. What we get is called the Klein bottle.

\textbf{Example:} Let $X = \mathbb{R}^{n+1} - 0$. We define an equivalence relation on $X$ by declaring $x \sim y$ if and only if $y = \alpha x$ for some $\alpha \in \mathbb{R}$. In words, we are saying that two points are equivalent if one is a scalar multiple of the other. So we see that the set of equivalence classes of $\sim$ is the same as the set of lines in $\mathbb{R}^n$ that pass through 0. We call the quotient space \textbf{projective space} and denote it $\mathbb{P}^n$.


Much like the product the quotient topology has a universal property. The idea here is that any map from $X$ to $Y$ that "respects the gluing" should give us a map from $X^*$ to $Y$.

\begin{theorem}[\textbf{Universal Property of the Quotient Topology}]
Let $q: X \rightarrow X^*$ be a quotient map. Let $f: X \rightarrow Y$ be a map that is constant on $q^{-1}(\{x'\})$ for all $x' \in X^*$. Then $f$ induces a natural map $f^*:X^* \rightarrow Y$ such that $f = f^* \circ q$.

\begin{center}
\begin{tikzcd}
X \arrow[r, "f"] \arrow[d, "q"'] & Y \\
X^* \arrow[ru, "f^*"', dashed]   &  
\end{tikzcd}
\end{center}

The map $f^*$ is continuous if and only if $f$ is continuous. 
\end{theorem}

The proof of this is one again a matter of unraveling the definitions so we will skip it here but it is very important to understand how the map $f^*$ is constructed. Pick a point $x' \in X^*$. Then since $q$ is surjective, we can pick a point $x \in X$ such that $q(x) = x'$. We define $f^*(x') = f(x)$. Now we may be multiple ways to pick $x$, but since $f$ is constant on $q^{-1}(\{x\})$, every choice will give us the same thing when we apply $f$.


The universal property is the most important thing to keep in mind when working with a quotient space.

\textbf{Example:} Let $X = [0,1]$ and let $X^*$ be the quotient space obtained by gluing 0 to 1. Let $f: X \rightarrow \mathbb{S}^1$ be the continuous map:

$$ f(x) = e^{2 \pi i x} = (\sin (2\pi x),\cos(2 \pi x)) $$

Then $f(0) = f(1)$ so $f$ passes to the quotient to give a map $f^*: X^* \rightarrow \mathbb{S}^1$. It is not hard to show that $f$ is bijective but showing it is an isomorphism is a bit harder. While we do have the tools to show this explicitly, we will hold off on showing it until later.

\textbf{Example:} Let $X = \mathbb{R}^n - 0$. We define an equivalence relation on $X$ by declaring that $x \sim y$ if and only if $y = \alpha x$ where $\alpha \in \mathbb{R}_{> 0}$. Now let $X^*$ be the quotient space. Note that this is different from projective space because we require $\alpha$ to be positive in the definition. So we are gluing along rays from the origin instead of lines through the origin. We will show that $X^*$ is homeomorphic to $\mathbb{S}^{n-1}$. First we define a map using the universal property of the quotient. Let $f: X \rightarrow \mathbb{S}^{n-1}$ by defined by

$$ f(x) = \frac{x}{|x|} $$

We denote the equivalence class of a point $x$ by $\overline{x}$. So to use the universal property of the quotient, we must show that $f$ is constant on $f^{-1}(\{\overline{x}\})$ for $\overline{x} \in X^*$. Let $y = \alpha x$. Then 

$$ \frac{y}{|y|} = \frac{\alpha x}{|\alpha x|} = \frac{\alpha x}{\alpha |x|} = \frac{x}{|x|} $$

The equality $|\alpha x| = \alpha |x|$ follows since $\alpha$ is positive.

So we obtain a map $f^*: X^* \rightarrow \mathbb{S}^{n-1}$.

To show that $f^*$ is a homeomorphism, we construct its inverse. Let $i: \mathbb{S}^{n-1}: \rightarrow X$ be the inclusion map. Let $g = q \circ i$. We will show that $f^*$ and $g$ are inverses.

$$f^*(g(x)) = f^*(\overline{x}) = \frac{x}{|x|} = x $$

$$ g(f^*(\overline{x})) = g\left( \frac{x}{|x|}\right) = \overline{\left( \frac{x}{|x|}\right) } = \overline{x}$$

\begin{theorem}
A bijective quotient map is a homeomorphism.
\end{theorem}

\begin{proof}
Let $f: X \rightarrow Y$ be a bijective quotient map. Since $f$ is bijective, the quotient map condition "$U$ is open if and only if $f^{-1}(U)$ is open" shows that $f^{-1}$ is a homeomorphism, therefore $f$ is as well.

\end{proof}

\textbf{Example:} Unfortunately, a quotient of a Hausdorff space is not always Hausdorff. For a classic example, let us look at the topological space $X = \{(x,y) \in \mathbb{R}^2 : y = 0,1 \}$. In other words, $X$ is two copies of $\mathbb{R}$. Now let's glue the two copies together everywhere except for at 0. We define $\sim$ by declaring $(x,0) \sim (x,1)$ for all $x \neq 0$. The quotient space $X^*$ is called the line with two origins. In your homework, you will show that all points of $X^*$ are closed but it still is not Hausdorff.

For an example of how bad things can go wrong, think about the quotient space of $\mathbb{R}$ obtained by gluing all the rational numbers together.


\section{Connected Spaces}

The idea behind connected spaces is very intuitive. If our topological space is two parallel lines, we'd like to say it is not connected. If our space is the x and y axes, we'd like to say it is connected. The definition is equally intuitive as we will see.

\begin{definition}
A \textbf{separation} of $X$ is a pair of nonempty open subsets of $X$, $U$ and $V$, such that $U \cup V = X$ and $U \cap V = \varnothing$. If a separation of $X$ does not exists, we call $X$ \textbf{connected}.
\end{definition}

\begin{theorem}
The following are equivalent:

\begin{enumerate}
    \item $X$ is not connected.
    \item There exists a nonempty proper subset of $X$ that is open and closed.
    \item $X = C \cup K$ where $C$ and $K$ are disjoint nonempty closed subsets of $X$.
    \item There exists a continuous surjection $f: X \rightarrow \{0,1\}$ (where $\{0,1\}$ has the discrete topology).
\end{enumerate}
\end{theorem}

\begin{proof}
$(1) \Rightarrow (2)$: If $X$ is not connected, then $X = U \cup V$ where $U \cap V = \varnothing$ and $U$ and $V$ are both nonempty. Since $X - U = V$, $U$ is both open and closed, proving that $(1) \Rightarrow (2)$.

$(2) \Rightarrow (3)$: Let $C$ be a nonempty proper subset of $X$ that is open and closed. Then $K = X - C$ is closed and $C \cup K = X$.

$(3) \Rightarrow (4)$: Let $X = C \cup K$ as in (3). Then the function $f:X \rightarrow \{0,1\}$ that maps $C$ to $0$ and $K$ to 1 is surjective and continuous: To see it is continuous, note that there are only four open subsets of $\{0,1\}$ and you should check that their preimages are open.

$(4) \Rightarrow (1)$: Let $f: X \rightarrow \{0,1\}$ be a surjective continuous map. Then $U = f^{-1}(\{0\})$ and $V = f^{-1}(\{1\})$ are a separation of $X$.

\end{proof}

We note here that it is usually pretty easy to show that a space is NOT connected, all we need to do is write down a separation. On the other hand, it is a bit harder to show a space is connected, because we have to somehow show it is impossible for a separation to exist. Thankfully, we do have a method of showing spaces are connected as we will see.

\begin{lemma} \label{lem}
Let $A$ be a connected subspace of $X$ and let $X = U \cup V$ be a separation of $X$. Then $A$ is contained in either $U$ or $V$.
\end{lemma}

\begin{proof}
$U \cap A$ and $V \cap A$ are disjoint open subsets of $A$. Since $A$ is connected, one of them must be empty, meaning $A$ is contained in the other.
\end{proof}

\begin{lemma} \label{lem2}
The union of a collection of connected subspaces of $X$ with a point in common is connected.
\end{lemma}
\begin{proof}
Let $\{U_\alpha\}$ be a collection of subspaces of $X$ and let $Y = \bigcup U_\alpha$. Let $p$ be the point that is in all of the $U_\alpha$. Now suppose, for the sake of contradiction, that $Y = C \cup D$ is a separation of $Y$. Then $p$ is in one of the two, so without loss of generality assume $p \in C$. Then for each $\alpha$, $U_\alpha \subseteq C$ by Lemma \ref{lem}. Thus, $D = \varnothing$ contradicting the assumption that $C$ and $D$ are a separation.
\end{proof}

We now have the ability to show a space is connected if we can break it down into smaller connected pieces, all of which intersect. Our next theorem is the most important of connectedness.

\begin{theorem}
The image of a connected space under a continuous map is connected.
\end{theorem}
\begin{proof}
Let $f: X \rightarrow Y$ be a continuous map. Then the restriction $f: X \rightarrow f(X)$ is continuous as well. Suppose, for the sake of contradiction, that $f(X) = U \cup V$ is a separation of $f(X)$. Then $X = f^{-1}(U) \cup f^{-1}(V)$ is a separation of $X$, contradicting the fact that $X$ is connected.

\end{proof}


\begin{theorem}
Let $X$ and $Y$ be connected topological spaces. Then $X \times Y$ is connected.
\end{theorem}
\begin{proof}
By Lemma \ref{lem2} it suffices to show that we can cover $X \times Y$ with connected subspaces that all have a point in common. Let $a \in X$. For any $b \in Y$, let 

$$ T_b = (\{a\} \times Y) \cup (X \times \{b\}) $$

$T_b$ should be thought of a cross, where we have a horizontal slice and a vertical slice, meeting at $(a,b)$. Since $\{a\} \times Y$ and $X \times \{b\}$ are homeomorphic to $Y$ and $X$ respectively, they are both connected. Since they share the point $(a,b)$, $T_b$ is connected by Lemma \ref{lem2}. 

Now we see that 

$$ X \times Y = \bigcup_{b \in Y} T_b $$

and all the $T_b$ have the point $(a,b)$ in common, so once again by Lemma \ref{lem2}, $X \times Y$ is connected.

\end{proof}

\section{Connected Subspaces of $\mathbb{R}$}

The main goal of this section is classifying connected subsets of $\mathbb{R}$ and using them to prove a generalized intermediate value theorem.

\begin{definition}
A subset $A \subseteq \mathbb{R}$ is \textbf{convex} if for all $a,b \in A$, $[a,b] \subseteq A$.
\end{definition}

\textbf{Exercise:} Show that $A$ is a convex subset of $\mathbb{R}$ if and only if $A$ is an interval or a ray.

Before we prove our main theorem, we recall the least upper bound property of $\mathbb{R}$. Given a set $A \subseteq \mathbb{R}$, a number $c \in \mathbb{R}$ is an \textbf{upper bound} of $A$ if for all $a \in A$, $a \leq c$. $c$ is called a \textbf{least upper bound} if $c$ is an upper bound and there are not smaller upper bounds. 

\begin{theorem}
Let $A \subseteq \mathbb{R}$. If $A$ has an upper bound then $A$ has a least upper bound, denoted $\sup A$.

\end{theorem}

\begin{theorem} \label{convex}
Let $Y$ be a subspace of $\mathbb{R}$. Then $Y$ is connected if and only if $Y$ is convex.
\end{theorem}
\begin{proof}
($\Rightarrow$) We do this direction by contrapositive so assume $Y$ is not convex. Then there exists, $a < c < b$ such that $a,b \in Y$ but $c \not \in Y$. Then

$$ Y = ((-\infty,c) \cap Y) \cup ((c, \infty) \cap Y) $$

is a separation of $Y$.

($\Leftarrow)$) Now we assume $Y$ is convex. Suppose for the sake of contradiction, that there exists a separation 

$$ Y = U' \cup V' $$

of $Y$. Pick an element $a \in U'$ and an element $b \in V'$. Without loss of generality assume $a < b$. Define

$$ U = [a,b] \cap U'  \hspace{1 cm} V = [a,b] \cap V' $$

Then $U$ and $V$ are a separation of $[a,b]$. Let $c = \sup U$. We will show that $c$ does not belong to $U$ or $V$, contradicting the fact that $[a,b] = U \cup V$.

\textit{Case 1: $c \in V$} In this case $c \neq a$, so since $V$ is open in $[a,b]$, we can find an $\epsilon > 0$ such that $(c- \epsilon, c] \subseteq V$. But $U$ and $V$ are disjoint which means $c- \frac{\epsilon}{2}$ is an upper bound for $U$, contradicting the fact that $c$ is the least upper bound of $U$.

\textit{Case 2: $c \in U$} In this case $c \neq b$. So since $U$ is open, we can find an $\epsilon > 0$ such that $[c,c+\epsilon) \subseteq U$. But $c + \frac{\epsilon}{2}$ is larger than $c$, contradicting the fact that $c$ is an upper bound of $U$.

\end{proof}

As an immediate corollary of this, we prove a generalized version of the intermediate value theorem from calculus. 

\begin{theorem}[\textbf{Intermediate Value Theorem}]
Let $X \rightarrow \mathbb{R}$ be a continuous function with $X$ connected. Let $x,y \in X$ and let $c$ be a real number lying between $f(x)$ and $f(y)$. Then there exists a point $z \in X$ such that $f(z) = c$.
\end{theorem}

\begin{proof}
By Theorem 11.5, $f(X)$ is connected so by Theorem \ref{convex}, $f(X)$ is convex.
\end{proof}

Now is a good time to introduce the notion of a path, which will be one of our main tools in this class.

\begin{definition}
Given $x,y \in X$, a \textbf{path} from $x$ to $y$ is a continuous function $\gamma: [a,b] \rightarrow X$ such that $\gamma(a) = x$ and $\gamma(b) = y$. $X$ is called \textbf{path connected} if every pair of points in $X$ have a path between them.
\end{definition}

\textbf{Example:} Let $a,b \in \mathbb{R}^n$. We define a the \textbf{straight line path} from $a$ to $b$ as $\gamma: [0,1] \rightarrow \mathbb{R}^n$:

$$ \gamma(t) = (1-t)a + tb $$

We see that $\gamma(0) = a$ and $\gamma(1) = b$ and we see it is continuous by looking at the coordinate functions. So $\mathbb{R}^n$ is path connected.

\begin{theorem}
If $X$ is path connected then $X$ is connected.
\end{theorem}
\begin{proof}
Suppose, for the sake of contradiction that $X$ is path connected and there exists a separation $X = U \cup V$. Pick points $x \in U$ and $y \in V$. Then there exists a path $\gamma: [a,b] \rightarrow X$ from $x$ to $y$. The image $\gamma[a,b]$ is connected by Theorem 11.5, so it must be contained entirely in $U$ or in $V$. But $\gamma(a) = x$ and $\gamma(b) = y$, a contradiction.
\end{proof}

\textbf{Example:} Let $X = \mathbb{R}^n \setminus 0$ with $n \geq 2$. For $a,b \in X$, we see that if the straight line path from $a$ to $b$ then $a$ and $b$ are scalar multiples of each other.

$$ (1-t)a + tb = 0 \Rightarrow a = \frac{t}{1-t} b $$

So we can pick a $c \in X$ that is not a scalar multiple of $a$ and $b$ and define a path piece-wise as the path from $a$ to $c$ then the path from $c$ to $b$. This shows that $X$ is path connected.

\begin{lemma}
The image of a path connected space under a continuous function is path connected.
\end{lemma}

\begin{proof}
Let $f: X \rightarrow Y$ be a continuous map with $X$ path connected. Let $f(x)$ and $f(y)$ be points in $f(X)$. Since $X$ is path connected, there exists a path $[a,b] \rightarrow X$ from $x$ to $y$. Then $f \circ \gamma$ is a path from $f(x)$ to $f(y)$.
\end{proof}

\textbf{Example:} The map $q: \mathbb{R}^{n+1}\setminus 0 \rightarrow \mathbb{S}^n$ defined by

$$ q(x) = \frac{x}{|x|} $$

is a continuous surjection (in fact it is a quotient map). So by the lemma, $\mathbb{S}^n$ is path connected for $n \geq 1$.

\textbf{Example:} This example is called the topologists sin curve. We define two subsets of $\mathbb{R}^2$:

$$ A = \{(0,y) : y \in [0,1]\} \hspace{1 cm} B = \left\{\left(x,\sin \left(\frac{1}{x}\right)\right): 0 < x \leq 1\right\} $$

Both $A$ and $B$ are connected since they are the images of continuous functions from connected spaces. Now we will show that $X = A \cup B$ is connected but not path connected. Suppose $X = U \cup V$ is a separation of $X$. Since $A$ and $B$ are connected they each must be entirely contained in $U$ or $V$. So the only possible separation of $X$ is $X = A \cup B$. Now $A$ and $B$ are disjoint so this is not immediately a contradiction. But we will show that $A$ is not open in $X$. This follows from the fact that any epsilon ball centered at the origin must intersect $B$.

To see a proof that $X$ is not path connected see page 157 of Munkres. 

\begin{definition}
Given $X$, define an equivalence relation by letting $x \sim y$ if there is a connected subspace of $X$ containing $x$ and $y$. The equivalence classes of $\sim$ are called the \textbf{connected components} of $X$. Doing the same thing but replacing connected with path connected gives the notion of \textbf{path components} of $X$.
\end{definition}

Showing that $\sim$ is actually an equivalence relation is a good exercise and uses some of the facts we've learned about connectedness.


\section{Compact Spaces}

We saw that the condition of being Hausdorff means having "enough" open sets to separate points. Compactness is a condition we can put on a space to say it has "not too many" open sets. We will see that compact Hausdorff spaces have just the right amount of open sets. But first we give some motivation.

Let $A$ be a subspace of $\mathbb{R}$. Since $A$ is contained in $\mathbb{R}$, it makes sense to ask if it is closed. It also makes sense to ask if it is bounded in $\mathbb{R}$. Now let $f: A \rightarrow \mathbb{R}$ be a continuous function. What can we say about the image of $f$ if we assume $A$ is closed or bounded?

We show that neither closed nor bounded are properties that are preserved under taking the image. Let $f: (0,1) \rightarrow \mathbb{R}$ be a homeomorphism. Then $f$ shows that the image of a bounded set need not be bounded and $f^{-1}$ shows that the image of a closed set need not be closed.

But something peculiar happens if we look at a set that is closed AND bounded, such as $[0,1]$. Try whatever continuous function you like, the image always seems to be either a point or another closed interval. Explaining this phenomena is compactness.

\begin{definition}
A collection $\mathcal{U}$ of subsets of $X$ is called a \textbf{cover} if the union of all the elements of $\mathcal{U}$ equals $X$. A cover is called an \textbf{open cover} if all of the elements are open subsets of $X$. Given a cover $\mathcal{U}$, a subcollection $\mathcal{A} \subseteq \mathcal{U}$ that is a cover of $X$ is called a \textbf{subcover} of $\mathcal{U}$.
\end{definition}

\textbf{Example:} For a topological space $X$, the topology on $X$, $\mathcal{T}$, is an open cover of $X$. The collection $\{X\}$ is a subcover of $\mathcal{T}$.

\textbf{Example:} Define $U_n \subseteq \mathbb{R}^n$ for $n \in \mathbb{N}$ by

$$ U_n = \{x \in \mathbb{R}^n : |x| < n \} $$

Then the collection $\mathcal{U} = \{U_n : n \in \mathbb{N} \}$ is an open cover of $\mathbb{R}^n$. The collection $\{U_n : \text{n is even}\}$ is a subcover of $\mathcal{U}$. The collection $\{U_1,U_2\}$ is not a subcover because it is not a cover of $X$.

\begin{definition}
A topological space $X$ is \textbf{compact} if every open cover of $X$ has a finite subcover.
\end{definition}

\textbf{Example:} Our last example shows that $\mathbb{R}^n$ is not compact. For any finite subset $\mathcal{A} \subseteq \mathcal{U}$, the union of all the elements of $\mathcal{A}$ is bounded, therefore not all of $\mathbb{R}^n$.

\textbf{Example:} Another proof that $\mathbb{R}$ is not compact is given by the following open cover:

$$ \mathcal{U} = \{(n, n+2) : n \in \mathbb{Z} \} $$

We see that $\mathcal{U}$ has no finite subcover since it has no proper subcovers at all since each integer is contained in only one element of $\mathcal{U}$.

\textbf{Example:} The interval $(0,1]$ is not compact as the following open cover shows:

$$  \mathcal{U} = \{(1/n, 1] : n \in \mathbb{N} \}  $$

Much like with connectedness, it is easier to show how to get compact spaces out of existing ones than it is to prove a space is compact from scratch. For this reason, we prove many properties of compact spaces here before giving examples.

Given a subspace $Y \subseteq X$, we say a collection of subsets $\mathcal{U}$ of $X$ \textbf{covers} $Y$ if $Y$ is contained in the union of all the elements of $\mathcal{U}$.

\begin{lemma}
Let $Y$ be a subspace of $X$. Then $Y$ is compact if and only if every collection of open subsets of $X$ that covers $Y$ contains a finite subcollection that covers $Y$.
\end{lemma}

\begin{proof}
The condition is the same as the definition of compactness once we recall that open subsets of $Y$ are the same as open subsets of $X$ intersected with $Y$.
\end{proof}


\begin{theorem} \label{closed}
Every closed subspace of a compact space is compact.
\end{theorem}
This will be proved as part of Homework 5.

\begin{theorem} \label{haus}
A compact subspace of a Hausdorff space is closed.
\end{theorem}

\begin{proof}
Let $Y$ be a compact subspace of a Hausdorff space $X$. Let $x_0 \in X - Y$. We will show that there is a neighborhood of $x_0$ disjoint from $Y$, proving that $Y$ is closed. For each $y \in Y$, we pick disjoint neighborhoods, $U_y$ and $V_y$ such that $x_0 \in U_y$ and $y \in V_y$. The collection of $U_y$ cover $Y$ so by compactness and lemma 13.3, we can pick finitely many of the $V_y$ that cover $Y$, label them $V_1,\dots,V_n$ and label the corresponding $U_y$ $U_1,\dots,U_n$. Now let 

$$ U = U_1 \cap \dots \cap U_n $$

Since there are finitely many, $U$ is open and since each $U_i$ contains $x$, $U$ contains $x$. We see that $U$ is disjoint from each $V_i$, so it is disjoint from their union. Their union covers $Y$, so $U$ is disjoint from $Y$.
\end{proof}

Much like connectedness, the property of being compact is miraculously preserved when taking images of continuous maps.

\begin{theorem} \label{imag}
The image of a compact space under a continuous map is compact.
\end{theorem}
\begin{proof}
Let $X$ be compact and let $f: X \rightarrow Y$ be a continuous map. Then the restricted map $f: X \rightarrow f(X)$ is continuous. Let $\mathcal{U}$ be an open cover of $f(X)$. Then 

$$ \{f^{-1}(U) : U \in \mathcal{U}\} $$

Is an open cover of $X$. Since $X$ is compact, we can pick finitely many of these open sets to cover $X$, label them $f^{-1}(U_1), \dots, f^{-1}(U_n)$. Then we see that the sets $U_1,\dots,U_n$ form an open subcover of $f(X)$.
\end{proof}

We now prove one of our more powerful theorems about compactness. We call a map $f: X \rightarrow Y$ \textbf{closed} if the image of a closed set is closed.

\begin{theorem}[\textbf{Closed Map Lemma}] \label{cml}
Let $X$ be a compact space and $Y$ a Hausdorff space. Then any continuous map $f: X \rightarrow Y$ is a closed map.
\end{theorem}

\begin{proof}
Let $K$ be a closed subset of $X$. Then $K$ is compact by Theorem \ref{closed}. Then $f(K)$ is compact by Theorem \ref{imag}. Since $Y$ is Hausdorff, Theorem \ref{haus} implies that $f(K)$ is closed.
\end{proof}

This lemma has an especially useful corollary.

\begin{corollary}
Let $f: X \rightarrow Y$ be a bijective continuous map from a compact space $X$ to a Hausdorff space $Y$. Then $f$ is a homeomorphism.
\end{corollary}

\begin{proof}
By Lemma \ref{cml}, $f$ is closed, which means for subset $K$ of $X$, $K$ is closed if and only if $f(K)$ is closed. Thus $f$ is a homeomorphism.
\end{proof}

I would now like to give you some examples of where compactness comes up in other areas of math. We will not be going into these any further, but they are here to show how important and wide reaching compactness is.

\textbf{Algebra:} Given a commutative ring $R$, we can put a topology on the prime ideals of $R$, Spec $R$, by declaring a set, $K$, of prime ideals to be closed if there exists an ideal $I$ of $R$ such that

$$ K = \{\mathfrak{p} \in \text{Spec } R : I \subseteq \mathfrak{p } \} $$

For any ring, this topology is compact.

\textbf{Model Theory/Logic:} Given a language $L$ (just a bunch of symbols), we can put a topology on the set of complete theories, $\mathfrak{T}$, (sentences with the symbols where every sentence is either true or false) by taking sets of the following form as a basis,

$$ B_\phi = \{ T \in \mathfrak{T} : \phi \in T \} $$

The fact that $\mathfrak{T}$ is compact with this topology is known as the Compactness Theorem and is one of the most important theorems in Model Theory.


\section{Compact Subspaces of $\mathbb{R}$}

In this section, we classify compact subspaces of $\mathbb{R}$.

\begin{theorem}
The closed interval $[0,1]$ is compact.
\end{theorem}

\begin{proof}
Let $\mathcal{U}$ be an open cover of $[0,1]$. We will prove it has a finite subcover. Let $A$ be the set of all points $x \in [0,1]$ such that $[0,x]$ can be covered by finitely many elements of $\mathcal{U}$. First, we see that $0 \in A$, since $0$ is in one of the elements of $\mathcal{U}$. So $A$ is nonempty. Since $A \subseteq [0,1]$, $A$ has an upper bound. So let $c = \sup A$.

First, we show that $c \neq 0$. Pick an element $U \in \mathcal{U}$ containing $0$. Since $U$ is open, there exists an $\epsilon > 0$ such that $[0,\epsilon) \subseteq U$. So $[0,\epsilon / 2)$ can be covered by a single element of $\mathcal{U}$, so $c > \epsilon / 2$.

Now we show that $c \in A$. Pick an element $V \in \mathcal{U}$ containing $c$. Then there exists an $\epsilon > 0$ such that $(c-\epsilon, c] \subseteq V$. Since $c$ is the supremum of $A$, there must be a point $y \in (c-\epsilon, c]$ such that $y \in A$. Thus the interval $[0,y]$ can be covered by finitely many $U_1,\dots,U_n \in \mathcal{U}$. So we see that $[0,c]$ is covered by $V,U_1,\dots,U_n$. Thus $c \in A$.

We finish the proof by showing that $c = 1$. Suppose, for the sake of contradiction that $c < 1$. Pick $V \in \mathcal{U}$ containing $c$. Then there exists an $\epsilon > 0$ such that $(c- \epsilon, c + \epsilon) \subseteq V$. Since $[0,c]$ can be covered by finitely many elements of $\mathcal{U}$, $[0,c + \epsilon / 2]$ can be covered by finitely many elements of $\mathcal{U}$ (just add $V$).

\end{proof}

Since any other closed interval $[a,b]$ is homeomorphic to $[0,1]$, we see that all closed intervals are compact.

Based on how hard we had to work to show that $[0,1]$ was compact, giving a full description of compact subsets of $\mathbb{R}$ might seem out of reach. Miraculously, we can do it and the hard part has already been taken care of by this one case.

\begin{theorem}
A subspace of $\mathbb{R}$ is compact if and only if it is closed and bounded.
\end{theorem}

\begin{proof}
First assume $Y$ is compact. Then $Y$ is closed in $\mathbb{R}$ since $\mathbb{R}$ is Hausdorff (Theorem 13.5). Now look at the covering

$$ \mathcal{U} = \{ (-n,n) : n \in \mathbb{N}\} $$

Since $Y$ is compact, it is contained in the union of finitely many of these (Lemma 13.3), therefore $Y$ is bounded.

Now assume $Y$ is closed and bounded. Since $Y$ is bounded, $Y \subseteq [-M,M]$ for some $M \in \mathbb{N}$. Since $Y$ is closed in $\mathbb{R}$, $Y$ is closed in $[-M,M]$. So $Y$ is a closed subspace of a compact space, therefore $Y$ is compact (Theorem 13.4).

\end{proof}

We can now prove a generalization of the extreme value theorem from calculus.

\begin{theorem}[\textbf{Extreme Value Theorem}]
Let $f: X \rightarrow \mathbb{R}$ be continuous with $X$ compact, then there exists $c,d \in X$ such that $f(c) \leq f(x) \leq f(d)$ for all $x \in X$.
\end{theorem}

\begin{proof}
Since $X$ is compact, $f(X)$ is compact, therefore it is closed and bounded. So let $a = \inf f(X)$ and $b = \sup f(X)$. Since $f(X)$ is closed, it contains all of its limit points, so $a,b \in f(X)$. Therefore we can find $c,d \in X$ such that $f(c) = a$ and $f(d) = b$. The theorem follows.
\end{proof}

As a special case of this we see that the image of a closed interval under a continuous map to $\mathbb{R}$ is either a closed interval or a point. This is because the closed interval is compact and connected, so the image must also be compact and connected. Since its a subset of $\mathbb{R}$, compact and connected is the same as being convex, closed and bounded which one can show is equivalent to being either a point or a closed interval. We leave some of the details out since we will not be using this fact later.


\section{The Product of Compact Spaces}
The purpose of this section is to prove that the product of compact spaces is compact. This will be our most involved proof so far, so we will break it up into smaller pieces. First, we have a technical result that is the key to the whole argument.

Consider the product space $X \times Y$. Let $U$ be an open subset of $X$. The set $U \times Y$ is called a \textbf{tube}.

\begin{lemma}[\textbf{The Tube Lemma}] \label{tube}
Consider the product space $X \times Y$ where $Y$ is compact and let $x_0 \in X$. If $W$ is an open subset of $X \times Y$ containing the slice $x_0 \times Y$, then there exists a neighborhood of $x_0$, $U \subseteq X$ such that the tube $U \times Y$ is contained in $W$.
\end{lemma}

\begin{proof}
Since $W$ is an open subset of $X \times Y$, for each $y \in Y$, we can find a basis element $U_y \times V_y$ that contains $(x_0,y)$. Let

$$ W' = \bigcup U_y \times V_y $$

Now the collection $\{U_y \times V_y\}$ covers $x_0 \times Y$, so by the compactness of $Y$, we can pick finitely many of them to cover $x_0 \times Y$, label them $U_1 \times V_1, \dots, U_n \times V_n$. Now let

$$ U = U_1 \cap \dots \cap U_n $$

Then $U$ is open since it is a finite intersection of open sets. Since $x_0 \in U_i$ for each $i$, $x_0 \in U$. Thus 

$$ U \times Y \subseteq W' \subseteq W $$
\end{proof}

We are now ready to prove our main theorem.

\begin{theorem}
Let $X$ and $Y$ be compact topological spaces. Then $X \times Y$ is compact.
\end{theorem}
\begin{proof}
Let $\mathcal{U}$ be an open cover of $X \times Y$. Consider a point $x_0 \in X$ and the slice $x_0 \times Y$. Since $x_0 \times Y$ is compact, we can find finitely many elements of $\mathcal{U}$ that cover $x_0 \times Y$, label them $W_1,\dots,W_n$. Let

$$ W = W_1 \cup \dots \cup W_n $$

Then $W$ contains all of $x_0 \times Y$. Therefore by the tube lemma, we can find a tube $U \times Y$ contained in $W$. We remember for later in the proof that $U \times Y$ is covered by finitely many elements of $\mathcal{U}$, namely the $W_i$.

Now we do this process for every $x \in X$ and obtain tubes $U_x \times Y$, containing $x \times Y$, each of which can be covered by finitely many elements of $\mathcal{U}$. We see that the $U_x$ cover $X$ and since $X$ is compact, we can pick finitely many to cover $X$, label them $U_1,\dots,U_n$. 

Now we have covered $X \times Y$ with finitely many tubes,

$$ X \times Y = (U_1 \times Y) \cup \dots \cup (U_n \times Y) $$

and each $U_i \times Y$ can be covered by finitely many elements of $\mathcal{U}$, so we are done.

\end{proof}

This lets us complete our goal of classifying compact subsets of $\mathbb{R}^n$. We follow the same proof as Theorem 14.2.

\begin{theorem}[\textbf{Heine-Borel Theorem}]
A subspace $Y \subseteq \mathbb{R}^n$ is compact if and only if it is closed and bounded.
\end{theorem}

\begin{proof}
First assume $Y$ is compact. Then $Y$ is covered by the sets

$$ U_n = \{x \in \mathbb{R}^n : |x| < n\} $$

where $n \in \mathbb{N}$. Since $Y$ is compact, $Y$ must be contained in the union of finitely many of these, therefore $Y$ is bounded. Since $\mathbb{R}^n$ is Hausdorff, $Y$ is closed by Theorem 13.5.

Now assume $Y$ is closed and bounded. Since $Y$ is bounded, $Y \subseteq [-M,M]^{n}$ for $M$ sufficiently large. Since $Y$ is closed in $\mathbb{R}^n$ and contained in $[-M,M]^{n}$, $Y$ is closed in $[-M,M]^{n}$.

The space $[-M,M]^{n}$ is compact by Theorem 14.1 and Theorem 15.2. So $Y$ is a closed subset of a compact space, therefore $Y$ is compact by Theorem 13.4.
\end{proof}

This lets us show that a whole heap of spaces are compact, such as the spheres $\mathbb{S}^n$, the torus $\mathbb{S}^1 \times \mathbb{S}^1$, the closed ball $B_1(0) = \{x \in \mathbb{R}^n : |x| \leq 1\}$ and even $\mathbb{P}^n$ since on your homework you constructed a surjective map $q: \mathbb{S}^n \rightarrow \mathbb{P}^n$.

Here is an application of Heine-Borel which does more than just show a certain space is compact. Recall that for points $x,y \in \mathbb{R}^n$, $d(x,y)$ is the distance between $x$ and $y$. Explicitely, it is 

$$ d(x,y) = \sqrt{ (x_1-y_1)^2 + \dots + (x_n-y_y)^2 } $$

Let $x_0 \in \mathbb{R}^n$ and let $A$ be a subset of $\mathbb{R}^n$. We define the distance between $x_0$ and $A$ to be 

$$ d(x_0, A) = \inf \{ d(x_0,a) : a \in A \} $$

\textbf{Example:} Consider the subspace $A = (0,1) \subseteq \mathbb{R}$. Then 

$$ d(2,A) = 1 $$

and yet we see that there is no point in $A$ that actually attains this distance.

\begin{theorem}
Let $A$ be a closed subspace of $\mathbb{R}^n$ and let $x_0$ be a point in $\mathbb{R}^n$. Then there exists a point $a_0 \in A$ such that 

$$ d(x_0,A) = d(x_0,a_0) $$

In other words, $a_0$ is a "closest point" to $A$.
\end{theorem}
Note that compactness is not mentioned anywhere in this theorem and yet it will be very useful in proving it.

\begin{proof}
Let $a$ be any point in $A$ and let $r = d(x_0,a)$. We consider the closed ball

$$ B_r(x_0) = \{ x \in \mathbb{R}^n : d(x_0,x) \leq d\} $$

We see that $a \in B_r(x_0)$, so the intersection of $B_r(x_0)$ and $A$ is nonempty. Let $Z = A \cap B_r(x_0)$. Since $A$ and $B_r(x_0)$ are both closed, $Z$ is closed and since $B_r(x_0)$ is bounded, $Z$ is bounded. Therefore $Z$ is compact by Heine-Borel.

Now we look at the continuous function $h: Z \rightarrow \mathbb{R}$ defined by

$$ h(z) = d(x_0,z) $$

Since $Z$ is compact, by the extreme value theorem, there exists a point $a_0 \in Z$ such that for all $z \in Z$

$$ d(x_0,a_0) \leq d(x,z) $$

So we have found a "closest point" among points of $Z$, now we show it is closest among all points of $A$. Let $x \in A - B_r(x_0)$. Then 

$$ d(x_0,x) > r = d(x_0,a) \geq d(x_0,a_0) $$

Thus $d(x_0,a_0) = d(x_0,A)$

\end{proof}


\section{Applications of Connectedness and Compactness}

Congratulations, you have officially finished the Point-Set Topology portion of this course! (Woo, yeeeaah)

Now is a good time to see some interesting applications of the tools we have developed, which will hopefully make you more comfortable with them.

Let's revisit our motivating question from earlier: Which topological spaces are homeomorphic to each other?

Now that we have the tool of connectedness, we can actually distinguish many spaces apart! There are the obvious examples of showing that connected things are not homeomorphic to not connected things, but somewhat surprisingly, this can even help us when both our spaces are connected as we will see here.

\begin{theorem}
$\mathbb{R}$ is not homeomorphic to $\mathbb{R}^n$ when $n \geq 2$.
\end{theorem}

\begin{proof}
Suppose, for the sake of contradiction, that there exists a homeomorphism $f: \mathbb{R} \rightarrow \mathbb{R}^n$. Then by restricting the domain and codomain, we see that $f_{\mathbb{R}\setminus 0}: \mathbb{R} \setminus 0 \rightarrow \mathbb{R}^n \setminus f(0)$ is a homeomorphism. But we showed in a previous class that $\mathbb{R}^n$ with a point taken out is connected, while $\mathbb{R} \setminus 0$ is not connected, a contradiction.
\end{proof}

This proof may lead you to believe we can do something similar to show that $\mathbb{R}^2$ and $\mathbb{R}^3$ are not homeomorphic by saying something like "if I remove a line from $\mathbb{R}^2$ it becomes disconnected while $\mathbb{R}^3$ minus a line is still connected". It is important to see why this proof does NOT work, as the image of a line under a continuous map may no longer be a line, it could be a whole plane for example. Once we develop the fundamental group, we will be able to show that $\mathbb{R}^2$ is not homeomorphic to $\mathbb{R}^n$ for $n \geq 3$.

On your homework, you will prove more spaces are not homeomorphic, as well as using connectedness to prove some very interesting "fixed point theorems".

Now let us give an application of compactness. A new example of a topological space that is of great interest in mathematics, especially in combinatorics, is polytopes. Polytopes are a natural generalization of polygons to higher dimensions.

Given a finite number of points $x_1,\dots,x_r$ in $\mathbb{R}^n$, we define the \textbf{polytope} defined by $x_1,\dots,x_r$, which we denote Conv$(x_1,\dots, x_r)$ to be the set

$$ \text{Conv}(x_1,\dots,x_r) = \left\{\epsilon_1 x_1 + \dots + \epsilon_r x_r : 0 \leq \epsilon_i \leq 1, \sum_{i=1}^r \epsilon_i = 1 \right\}$$

Admittedly it is probably better to define the polytope to be the smallest convex set that contains $x_1,\dots,x_n$ then show that our definition is equivalent.

The reason we defined polytopes the way we did is that we can now apply our tools to show that all polytopes are compact.

\begin{theorem}
Polytopes are compact.
\end{theorem}

\begin{proof}
Let $x_1,\dots,x_r \in \mathbb{R}^n$. Consider the space $[0,1]^r$ (the product with itself $r$ times). Then the subset

$$ K = \left\{(\epsilon_1,\dots,\epsilon_r) \in [0,1]^r : \sum_{i=1}^r \epsilon_i = 1\right\} $$

is closed because it is the preimage of the set containing 1 under the continuous function $f: [0,1]^r \rightarrow \mathbb{R}$ defined by

$$ f(\epsilon_1,\dots,\epsilon_r) = \epsilon_1+\dots+\epsilon_r $$

Since $K$ is a closed subspace of a compact space, $K$ is closed.

Now we define a continuous function $g: K \rightarrow \mathbb{R}^n$ by

$$ g(\epsilon_1,\dots,\epsilon_r) = \epsilon_1 x_1 + \dots + \epsilon_r $$

To see that $g$ is continuous, note that it is the restriction of a continuous function from $\mathbb{R}^n$ to $\mathbb{R}$ to the subspace $K \subseteq \mathbb{R}^n$.

The image of $g$ is Conv$(x_1,\dots,x_r)$ by definition and the image of a compact set is compact, therefore Conv$(x_1,\dots,x_r)$ is compact.

\end{proof}


One example of a way these tools can be used is for the famous Bolzano Weierstrass theorem.

\begin{theorem}[\textbf{Bolzano-Weierstrass}]
Any infinite bounded subset of $\mathbb{R}^n$ has a limit point.
\end{theorem}

There are various forms of this theorem, some state that any bounded sequence of points in $\mathbb{R}^n$ has a convergent subsequence. These can be shown to be equivalent so we just stick with the version we stated.

Any bounded subset of $\mathbb{R}^n$ is a subset of $[-M,M]$ for $M$ large enough so it suffices to prove the following theorem.

\begin{theorem}
Let $X$ be a compact space. Every infinite subset of $X$ has a limit point.
\end{theorem}

\begin{proof}
Let $A$ be a subset of $X$ with no limit points. Then $A$ is closed, because it trivially contains all of it's limit points, therefore $A$ is compact as a subspace of $X$. Since $A$ has no limit points, for each $a \in A$, there is an open subset $U_a$ of $X$ that intersects $A$ only at $a$. Therefore, $A$ has the discrete topology because each point in $A$ is open. It is easy to check that a set with the discrete topology is compact if and only if the set is finite.
\end{proof}


\section{Crash Course in Group Theory}

Before we talk about the fundamental group, we will introduce what a group is abstractly and give come concrete examples.

A \textbf{binary operation} on a set $G$ is a function $\ast: G \times G \rightarrow G$. In the context of group theory, we will write $a \ast b$ instead of $\ast(a,b)$.

\begin{definition}
A group $(G, \ast)$ is a set $G$ along with a binary operation $\ast$ on $G$ that satisfies the following three properties:

\begin{enumerate}
    \item (Associativity) $(a \ast b) \ast c = a \ast (b \ast c)$ for all $a,b,c \in G$.
    \item There exists an element $e \in G$ such that $a \ast e = e \ast a = a$ for all $a \in G$. $e$ is called the \textbf{identity} of $G$.
    \item For each $a \in G$, there exists an element $a^{-1} \in G$ such that $a \ast a^{-1} = a^{-1} \ast a = e$. $a^{-1}$ is called the \textbf{inverse} of $a$.
\end{enumerate}
\end{definition}

You may have noticed that in the definition, we did not require the identity to be unique and yet we called it \textit{the} identity. We also did the same thing with inverses. To justify that, we prove that the identity and inverses are actually unique.

\begin{proposition}
Let $(G,\ast)$ be a group. There is a unique identity in $G$ and every element $a \in G$ has a unique inverse.
\end{proposition}

\begin{proof}
The existence of an identity follows from axiom 2 of being a group so suppose $e_1$ and $e_2$ are both identities. Then,

$$ e_1 = e_1 \ast e_2 = e_2 $$

by axiom 2. We now do a similar argument for inverses.

Let $a \in G$. By axiom 3, $a$ has an inverse, so suppose $b$ and $c$ are both inverses of $a$. Then

$$ b = b \ast e = b \ast (a \ast c) = (b \ast a) \ast c = e \ast c = c $$

\end{proof}

Now that we have defined what a group is, let's look at some examples.

\textbf{Example:} $(\mathbb{Z},+)$ is a group. Let us check the axioms, first

$$ (a+b)+c = a+(b+c) $$

We see that $0$ is the identity because

$$ a+0 = a+0 = a $$

For each $a \in \mathbb{Z}$, $-a$ is the inverse of $a$,

$$ a + (-a) = (-a) + a = 0 $$

As you can see, even though the axioms for being a group may look complicated, it is usually pretty straightforward to check that something is a group. For this reason, for the next few examples, rather than write all the details out again, we will just note what the identity and inverses are.

\textbf{Example:} $(\mathbb{R},+)$ is a group for the same reason as $(\mathbb{Z},+)$, $0$ is the identity and $-a$ is the inverse of $a$.

\textbf{Example:} $(\mathbb{R} \setminus 0, \cdot)$ is a group where $\cdot$ denotes multiplication. We see that $1$ is the identity in this group and $1/a$ is the inverse of $a$.

\textbf{Example:} The one point set $\{e\}$ with the trivial group law of $e \ast e = e$ forms a group called the \textbf{trivial group}. This is the only binary relation on $\{e\}$ so it is the only possible group with one element.

\textbf{Example:} $(\mathbb{C},+)$ and $(\mathbb{C} \setminus 0,\cdot)$ are groups for the same reason as we saw above with $\mathbb{R}$.

Just like with topological spaces, we would like a way of saying if two groups are "the same". With topological spaces, we defined continuous maps, which were maps that preserve the topological structure that we put on our spaces, then defined homeomorphisms using them. We do the same for groups.

\begin{definition}
A map $\phi: G \rightarrow H$ between groups $(G, \ast)$ and $(H,\cdot)$ is called a \textbf{homomorphism} if for all $a,b \in G$,

$$ \phi(a \ast b) = \phi (a) \cdot \phi (b) $$

A bijective homomorphism is called an \textbf{isomorphism}.
\end{definition}

We will use the notation $\phi: (G, \ast) \rightarrow (H, \cdot)$ to denote a homomorphism to make the group operations clear.

\begin{lemma}
Let $f: (G,\ast) \rightarrow (H,\cdot)$ be an isomorphism. Then $f^{-1}: (H,\cdot) \rightarrow (G,\ast)$ is an isomorphism.
\end{lemma}

This lemma is saying that the awkward thing that happened with topological spaces, where we could have a bijective continuous map that was not a homeomorphism, does not happen with groups. The proof will be in your homework.

\textbf{Example:} The map $\phi: (\mathbb{Z},+) \rightarrow (\mathbb{Z},+)$ defined by $\phi(a) = 2x$ is a homomorphism,

$$ \phi(a + b) = 2(a + b) = 2a + 2b = \phi(a) + \phi(b) $$

\textbf{Example:} The map $i: (\mathbb{Z},+) \rightarrow (\mathbb{R},+)$ defined by $i(a) = a$ is a homomorphism.

\textbf{Example:} The map $\psi: (\mathbb{R},+) \rightarrow (\mathbb{R}_{>0},\cdot)$ defined by $\psi(x) = e^x$ is an isomorphism,

$$ \psi(x+y) = e^{x+y} = e^x \cdot e^y = \psi(x) \cdot \psi(y) $$

\begin{theorem}[\textbf{Properties of Homomorphisms}]
Let $\phi: (G,\ast) \rightarrow (H,\cdot)$ be a homomorphism.

(a) $\phi$ maps the identity of $G$ to the identity of $H$.

(b) For all $a \in G$, $\phi(a^{-1}) = \phi(a)^{-1}$

(c) Let $\psi: (H,\cdot) \rightarrow (J,\#)$ be a homomorphism, then $\psi \circ \phi$ is a homomorphism.

\end{theorem}

\begin{proof}
(a) Let $e_G$ and $e_H$ be the identities of $G$ and $H$ respectively. First, we see that 

$$\phi(e_G) = \phi(e_G \ast e_G) = \phi(e_G) \cdot \phi(e_G) $$

Now we take both sides and apply $\cdot \phi(e_G)^{-1}$ and we obtain $e_H = \phi(e_G)$.

(b) Let $a \in G$. Then

$$ \phi(a) \cdot \phi(a^{-1}) = \phi(a \ast a^{-1}) = \phi(e_G) = e_H $$

Since inverses are unique by Proposition 16.2, we get $\phi(a^{-1}) = \phi(a)^{-1}$.

(c) This is just an explicit calculation,

$$ \psi(\phi(a \ast b)) = \psi(\phi(a) \cdot \phi(b)) = \psi(\phi(a)) \# \psi(\phi(b)) $$

\end{proof}


\section{Homotopy}
A homotopy is, intuitively speaking, a continuous deformation of one map to another. From now on we will fix the notation that $I = [0,1]$ because this space is going to come up so often.

\begin{definition}
Let $f: X \rightarrow Y$ and $g: X \rightarrow Y$ be continuous maps. A \textbf{homotopy} between $f$ and $g$ is a continuous map $H: X \times I \rightarrow Y$ such that 

\begin{center}
$H(x,0) = f(x)$ \hspace{0.2cm}  and  \hspace{0.2cm} $H(x,1) = g(x)$
\end{center}
\end{definition}

Notice that for any $t_0 \in I$, we obtain a map from $X$ to $Y$ by fixing the second variable of $H$,

$$ H(x,t_0) $$

So we get a family of functions from $X$ to $Y$, one for each element of $I$. This is why we think of a homotopy as "continuously deforming" one map into another.

If there exists a homotopy between $f$ and $g$, we say $f$ and $g$ are \textbf{homotopic} and write $f \simeq g$.

\begin{theorem}
Let $f:X \rightarrow \mathbb{R}^n$ and $g: X \rightarrow \mathbb{R}^n$ be any two maps. Then $f$ and $g$ are homotopic.
\end{theorem}

\begin{proof}
We define an explicit homotopy $H: X \times I \rightarrow \mathbb{R}^n$ by

$$ H(x,t) = (1-t)f(x) + tg(x) $$

It is easy to see that $H(x,0) = f(x)$ and $H(x,1) = g(x)$ but how do we show that $H$ is continuous. To do this we prove that $H$ can be written as the composition of continuous functions. First we define maps $\Delta: X \rightarrow X \times X$ and $S: \mathbb{R}^n \times \mathbb{R}^n \times I \rightarrow \mathbb{R}^n$ by,

$$ \Delta(x) = (x,x) \text{\hspace{1cm}}  S(x,y,t) = (1-t)x + ty $$

Then $\Delta$ is continuous by the universal property of the product and $S$ is continuous by calculus. Now we see that we can write $H$ as the following composition (recall the product of two maps from Homework 3),

\begin{center}

\begin{tikzcd}
X \times I \arrow[rr, "\Delta \times id_I"] \arrow[rrrrr, "H", bend right] &  & X \times X \times I \arrow[rr, "f \times g \times id_I"] &  & \mathbb{R}^n \times \mathbb{R}^n \times I \arrow[r, "S"] & \mathbb{R}^n
\end{tikzcd}

\end{center}

thus proving that $H$ is continuous. 

\end{proof}

We call the homotopy constructed in this proof the \textbf{straight line homotopy} from $f$ to $g$ and it will be the most important homotopy that we have.

We now consider the special case where our maps are paths, $f: I \rightarrow X$ and $g: I \rightarrow X$. Let $x_0 = f(0)$ and $x_1 = f(1)$. We call $x_0$ the \textbf{initial} point of $f$ and $x_1$ the \textbf{final} point of $f$. We also say $f$ is a path \textit{from $x_0$ to $x_1$}.

If $f$ and $g$ are paths with the same initial and final points, we would like to talk about a stronger version of homotopy, where the initial and final points stay fixed during the homotopy.

\begin{definition}
Let $f: I \rightarrow X$ and $g: I \rightarrow X$ be two paths from $x_0$ to $x_1$. A \textbf{path homotopy} between $f$ and $g$ is a homotopy $H: I \times I \rightarrow X$ such that for all $t \in I$,

\begin{center}
$ H(0,t) =  x_0 $ \hspace{0.2cm} and $\hspace{0.2cm} H(1,t) = x_1 $
\end{center}

\end{definition}

Because $H$ is also a homotopy, we have that

\begin{center}
$ H(s,0) =  f(s) $ \hspace{0.2cm} and $\hspace{0.2cm} H(s,1) = g(s) $
\end{center}

Keeping track of which copy of $I$ does what can be a little confusing, so we make sure to always put the path variable in the first coordinate and denote it with an $s$ and put homotopy variable in the second coordinate and denote it with a $t$.

The definition of a path homotopy implies that for each $t_0 \in I$, we obtain a path from $x_0$ to $x_1$ by fixing the second variable,

$$ H(s,t_0) $$

If there exists a path homotopy between paths $f$ and $g$, then we say that $f$ and $g$ are \textbf{path homotopic} and write $f \simeq_p g$.

\begin{theorem}
Let $x_0,x_1 \in \mathbb{R}^n$. Let $f$ and $g$ be two paths from $x_0$ to $x_1$. Then $f$ and $g$ are path homotopic.
\end{theorem}

\begin{proof}
Let $H: I \times I \rightarrow \mathbb{R}^n$ be the straight line homotopy from $f$ to $g$. Then

$$ H(0,t) = (1-t)f(0) + tg(0) = (1-t)x_0 + t x_0 = x_0$$

$$ H(1,t) = (1-t)f(1) + tg(1) = (1-t)x_1 + t x_1 = x_1$$

Thus $H$ is a path homotopy.
\end{proof}

So far we have not seen any paths that are not path homotopic. Proving that this can happen is actually a very hard problem and is basically what we are going to spend the next week proving, but let's see an example where it at least intuitively feels like it happens.

\textbf{Example:} Let us look at two paths from $(1,0)$ to $(-1,0)$ in $\mathbb{R}^2 \setminus 0$ defined by

$$ f(x) = (\cos x, \sin x) $$
$$ g(x) = (\cos x, -\sin x) $$

We see that $f$ goes "above" 0 and $g$ goes "below" 0 and so it should be impossible for these two paths to be path homotopic, as they are paths into $\mathbb{R}^2 \setminus 0$ and any path homotopy between them would surly pass through 0. This should feel somewhat like a 2 dimensional version of the intermediate value theorem. This intuition is correct but we don't quite have the tools to prove it yet.

Now that we have an intuitive idea as to how homotopies work, let's dive into the weeds and actually start proving things about them.

\begin{theorem}
The relations $\sim$ and $\sim_p$ are equivalence relations.
\end{theorem}

\begin{proof}
Let $f: X \rightarrow Y$ be a continuous map. Then $H: X \times I \rightarrow Y$ defined by

$$ H(x,t) = f(x) $$

is a homotopy from $f$ to $f$. So $f \sim f$. If $f$ is a path, $H$ is a path homotopy. So $f \sim_p f$ in that case.

Now suppose $f \sim g$. Let $H: X \times I \rightarrow Y$ be a homotopy from $f$ to $g$. Then $G: X \times I \rightarrow Y$ defined by 

$$ G(x,t) = H(x,1-t) $$

is a homotopy from $g$ to $f$, so $g \sim f$. If $f$ and $g$ are paths and $H$ is a path homotopy, then $G$ is a path homotopy as well.

Now suppose $f \sim g$ and $g \sim j$. Let $H_1: X \times I \rightarrow Y$ and $H_2: X \times I \rightarrow Y$ be homotopies from $f$ to $g$ and $g$ to $j$ respectively. We define a new homotopy $G: X \times I \rightarrow Y$ by

\[ G(x,t) =  \left\{
\begin{array}{ll}
      H_1(x,2t) & 0 \leq t \leq \frac{1}{2} \\
      H_2(x,2t-1) & \frac{1}{2} \leq t \leq 1 \\
\end{array} 
\right. \]

We see that $G$ is well defined because for $t = \frac{1}{2}$, we have

$$ G\left(x,\frac{1}{2}\right) = H_1(x,1) = H_2(x,0) = g(x) $$

so by the gluing lemma, $G$ is continuous. It it clear that $G$ is a homotopy from $f$ to $j$ and once again you can check that it is a path homotopy if we assume everything is a path.
\end{proof}

We will denote the equivalence class (the equivalence relation is $\sim_p$) of the path $f$ by $[f]$. If $f$ is a path, we will always be working with the equivalence relation $\sim_p$ not $\sim$.

Let $x_0,x_1,x_2 \in X$. Suppose we have a path $f$ from $x_0$ to $x_1$ and a path $g$ from $x_1$ to $x_2$. Now we define a path $f \ast g$ from $x_0$ to $x_2$ called the \textbf{product} of $f$ and $g$ by,

\[ (f \ast g)(s) =  \left\{
\begin{array}{ll}
      f(2s) & 0 \leq s \leq \frac{1}{2} \\
      g(2s-1) & \frac{1}{2} \leq s \leq 1 \\
\end{array} 
\right. \]

Since $f(1) = g(0) = x_1$, $f \ast g$ is a well defined continuous function and we see that $(f \ast g)(0) = f(0) = x_0$ and $(f \ast g)(1) = g(1) = x_2$ so it is indeed a path from $x_0$ to $x_2$.

We can hope to extend the product to equivalence classes of path with the formula

$$ [f] \ast [g] = [f \ast g] $$

For this to make sense, we have to prove that if we pick different representatives for the equivalence classes of $[f]$ and $[g]$ that we get the same thing. This is made precise by the following lemma.

\begin{lemma}
Let $f$ and $f'$ be path homotopic and let $g$ and $g'$ be path homotopic. Then $f \ast g$ is path homotopic to $f' \ast g'$.
\end{lemma}

\begin{proof}
Let $F$ and $G$ be path homotopies from $f$ to $g$ and $f'$ to $g'$ respectively. We construct a homotopy $H$ from $f \ast g$ to $f' \ast g$ by

\[ H(x,t) =  \left\{
\begin{array}{ll}
      F(2s,t) & 0 \leq t \leq \frac{1}{2} \\
      G(2s-1,t) & \frac{1}{2} \leq t \leq 1 \\
\end{array} 
\right. \]

We leave it to you to show $H$ is well-defined and indeed a path homotopy.

\end{proof}

Now that we have proven it is well defined, we will officially define the product of two equivalence classes of paths by

$$ [f] \ast [g] = [f \ast g] $$

Finally, we present a list of properties about $\ast$ which will be very important going foreword. Understanding why each of these is true is very important and I implore you to draw pictures and think about how to prove each of these by constructing homotopies like we have done above. For the sake of time, however, we will be skipping this proof, to read it look on page 327 of Munkres.

\begin{theorem}
In the list that follows, whenever we write $[f] \ast [g]$, it is implied that $f$ and $g$ are paths and the final point of $f$ is the initial point of $g$.

\begin{enumerate}
    \item (Associativity) $([f] \ast [g]) \ast (h) = [f] \ast ([g] \ast [h])$
    \item (Identities) Let $f$ be a path from $x_0$ to $x_1$. Let $e_0: I \rightarrow X$ be the constant path that maps all of $I$ to $x_0$. Define $e_1$ the same way for $x_1$. Then
    $$ [e_0] \ast [f] = [f] \text{\hspace{1cm}}  [f] \ast [e_1] = [f] $$
    \item (Inverses) Let $f: I \rightarrow X$ be a path in $X$ from $x_0$ to $x_1$. We define $\overline{f}: I \rightarrow X$ by $\overline{f}(s) = 1-s$. Then
    $$ [f] \ast [\overline{f}] =  [e_0] \text{\hspace{1cm}}  [\overline{f}] \ast [f] = [e_1] $$
\end{enumerate}
\end{theorem}

It is important to note that these are all equalities of equivalence classes. So if we look at $f \ast \overline{f}$ for example. This path is not equal to the constant path $e_0$, it is only homotopic to it. As a path $f \ast \overline{f}$ looks like following $f$, then doing $f$ in reverse.


\section{The Fundamental Group}
In this section we define the fundamental group and prove some important properties about it. Much like connectedness and compactness, it is easier to prove things about the fundamental group than it is to actually give examples of it.

\begin{definition}
Let $X$ be a topological space and let $x_0$ be a point of $X$. A path $f: I \rightarrow X$ from $x_0$ to $x_0$ is called a loop based at $x_0$.
\end{definition}

\begin{definition}
Let $X$ be a topological space and let $x_0$ be a point of $X$. The set of path homotopy classes of loops based at $x_0$ is called the \textbf{fundamental group} of $X$ with base point $x_0$ and is denoted $\pi(X,x_0)$. 
\end{definition}

So the elements of $\pi(X,x_0)$ are of the form $[f]$ where $f$ is a function from $I$ to $X$ such that $f(0)=f(1)=x_0$ and two elements $[f]$ and $[g]$ are the same if and only if $f$ and $g$ are path homotopic.

We notice that if $f$ and $g$ are two loops based at $x_0$ then $f \ast g$ is a loop based at $x_0$. 

\begin{theorem}
$(\pi(X,x_0),\ast) $ is a group.
\end{theorem}

\begin{proof}
This follows directly from the properties in Theorem 18.7. Associativity is part (1). Part (2) tells us that the constant path $e_{x_0}$ is the identity of our group and part (3) says that for any loop $f$, the reverse loops $\overline{f}$ gives an inverse of $f$.
\end{proof}

As you can see, the notation is a bit cumbersome so from this point onward, we will just write $\pi(X,x_0)$ when talking about the group and leave out the $\ast$ from the notation. From this point on, we will also use Theorem 18.7 freely without directly citing it.

The first thing we would like to prove is that if $X$ is path connected, then the fundamental group does not depend on which base point we pick.

Let $\alpha$ be a path between $x_0$ and $x_1$ in $X$. Then we can define a map 

$$ \hat{\alpha}: \pi(X,x_0) \rightarrow \pi(X,x_1) $$

by the equation

$$ \hat{\alpha} ([f]) = [\overline{\alpha}] \ast [f] \ast [\alpha] $$

\begin{theorem}
The map $\hat{\alpha}$ is an isomorphism.
\end{theorem}

\begin{proof}
To show $\hat{\alpha}$ is an isomorphism, we must show it is a homomorphism and that it has an inverse. First we show it is a homomorphism. Let $f$ and $g$ be loops based at $x_0$. Then

$$ \hat{\alpha}([f]) \ast \hat{\alpha}([g]) = [\overline{\alpha}] \ast [f] \ast [\alpha] \ast [\overline{\alpha}] \ast [g] \ast [\alpha]  $$
$$ = [\overline{\alpha}] \ast [f] \ast [g] \ast [\alpha]  $$
$$ = \hat{\alpha}([f] \ast [g]) $$

Now we show $\hat{\alpha}$ has an inverse. Let $\beta$ denote $\overline{\alpha}$. We show $\hat{\beta}: \pi(X,x_1) \rightarrow \pi(X,x_0)$ is the inverse of $\hat{\alpha}$. First let us compute what $\hat{\beta}$ does to elements of $\pi(X,x_1)$.

$$ \hat{\beta}([f]) = [\overline{\beta}] \ast [f] \ast [\beta] = [\alpha] \ast [f] \ast [\overline{\alpha}] $$

Now we show $\hat{\beta}$ is the inverse of $\hat{\alpha}$.

$$ \hat{\alpha}(\hat{\beta}([f])) = [\overline{\alpha}] \ast ([\alpha] \ast [f] \ast [\overline{\alpha}]) \ast [\alpha] = [f] $$

The same calculation shows that $\hat{\beta}(\hat{\alpha}([h])) = [h]$.

\end{proof}

\begin{corollary}
If $X$ is path connected and $x_0$ and $x_1$ are points in $X$ then $\pi(X,x_0)$ and $\pi(X,x_1)$ are isomorphic.
\end{corollary}

\begin{proof}
Pick a path $\alpha$ from $x_0$ to $x_1$. Then $\hat{\alpha}$ is an isomorphism from $\pi(X,x_0)$ to $\pi(X,x_1)$.
\end{proof}

\begin{definition}
A space $X$ is called \textbf{simply connected} if $X$ is path connected and for some $x_0 \in X$, $\pi(X,x_0)$ has only one element.

\end{definition}

\textbf{Example:} By Theorem 18.4, any two loops in $\mathbb{R}^n$ with the same base point are path homotopic. Therefore $\pi(X,x_0)$ has only one element for all $x_0 \in \mathbb{R}^n$. Since $\mathbb{R}^n$ is also path connected, $\mathbb{R}^n$ is simply connected.

In Homework 7, you will prove a relatively simple criteria for being simply connected. Later we will prove that $\mathbb{S}^n$ is simply connected for all $n \geq 2$.

Now we show that a continuous map of topological spaces induces a homomorphism of their fundamental groups. Because fundamental groups need base points, we introduce the following notation to keep track of them. We say 

$$ f: (X,x_0) \rightarrow (Y,y_0) $$

is continuous if $f$ is a continuous function from $X$ to $Y$ and $f(x_0) = y_0$.

\begin{definition}
Let $h: (X,x_0) \rightarrow (Y,y_0)$ be continuous. We define a map $h_*: \pi(X,x_0) \rightarrow \pi(Y,y_0)$ by

$$ h_*([f]) = [h \circ f] $$

We call $h$ the \textbf{homomorphism induced by $h$}. 

\end{definition}

First, because we are working with equivalence classes, we need to argue that $h_*$ is actually well defined. Let $[f] = [g]$. Then there exists a path homotopy $H: I \times I \rightarrow X$ from $f$ to $g$. Then $h \circ H$ is a path homotopy from $h \circ f$ to $h \circ g$. Thus $h_*$ is well defined. To see that $h_*$ is a homomorphism, verify the following equation

$$ (h \circ f) \ast (h \circ g) = h \circ (f \ast g) $$

We get actual equality here, not just that there are homotopic, you just need to write out both sides.

Now we prove that this process is \textit{functorial}. If you've never seen that word before, do not be alarmed, just know that the following property is very important and will show up many more times with many other mathematical objects down the line.

\begin{theorem}
If $h: (X,x_0) \rightarrow (Y,y_0)$ and $k: (Y,y_0) \rightarrow (Z,z_0)$ are continuous, then $(k \circ h)_* = k_* \circ h_*$. If $i: (X,x_0) \rightarrow (X,x_0)$ is the identity map then $i_*: \pi(X,x_0) \rightarrow \pi(X,x_0)$ is the identity map.
\end{theorem}

\begin{proof}
To prove two maps are equal, you just show that they are equal when evaluating where elements go,

$$ (k \circ h)_* ([f]) = [k \circ h \circ f] = (k_* \circ h_*)([f]) $$

and

$$ i_*([f]) = [i \circ f] = [f] $$

\end{proof}

This lets us give a short proof that the fundamental group is a topological invariant (kind of, it still depends on choosing a base point first).

\begin{corollary}
Let $h: (X,x_0) \rightarrow (Y,y_0)$ be a homeomorphism. Then $h_*$ is an isomorphism.
\end{corollary}

\begin{proof}
Let $j$ be the inverse of $h$. Then $j_* \circ h_* = (j \circ h)_* = (id_X)_*$ and $h_* \circ j_* = (h \circ j)_* = (id_Y)_*$. Thus $j_*$ is the inverse of $h_*$ and therefore they are both isomorphisms.

\end{proof}

Using this you can prove that being simply connected is a topological property.


\section{The Fundamental Group of the Circle}

In this section, we compute the fundamental group of the circle. This will be our one and only explicit computation of a fundamental group and while the end result may seem obvious, we will give numerous applications to show how important it is. Despite how simple it is intuitively, this will be the most technical proof we do all quarter. Our proof would potentially be more clear with the language of covering spaces but since we have already introduced a lot of new concepts quickly, we instead just work specifically with $\mathbb{S}^1$.

First we recall a map that we have seen before. Let $p: \mathbb{R} \rightarrow \mathbb{S}^1$ be defined by

$$ p(t) = (\cos(2\pi t), \sin(2 \pi t) ) $$

This map will be the key to our proof so we first prove an important property that it has.

\begin{theorem}
For each $x \in \mathbb{S}^1$, there exists a neighborhood $U$ of $x$ such that $p^{-1}(U)$ can be written as a disjoint union

$$ p^{-1}(U) = \dots \overset{\cdot}{\cup} V_{-1} \overset{\cdot}{\cup} V_0 \overset{\cdot}{\cup} V_1 \overset{\cdot}{\cup} \dots  $$

such that each $V_i$ is an open interval and $p$ restricted to $V_i$ is a homeomorphism onto $U$.
\end{theorem}

\begin{proof}
Let us start with an example. Let $U$ be the "open right half" of the circle. Explicitly, $U$ is the subset of $\mathbb{S}^1$ where the first coordinate is positive. Then $p^{-1}(U)$ are the points $t \in \mathbb{R}$ for which $\cos(2\pi t)$ is positive. By basic properties of $\cos$, we see that

$$ p^{-1}(U) = \dots \overset{\cdot}{\cup} V_{-1} \overset{\cdot}{\cup} V_0 \overset{\cdot}{\cup} V_1 \overset{\cdot}{\cup} \dots $$

where $V_n = (n- \frac{1}{4}, n + \frac{1}{4})$

Now we would just like to show that $p$ maps $V_n$ homeomorphically onto $U$. We see that $p$ is injective on $\overline{V_n} = [n- \frac{1}{4}, n + \frac{1}{4}]$, therefore, by compactness, $p$ restricted to $\overline{V_n}$ is a homeomorphism onto its image. In particular, $p$ restricted to $V_n$ is a homeomorphism onto its image, which is $U$.

So for all points of $\mathbb{S}^1$ with positive first coordinate, we have shown that $U$ gives a neighborhood that satisfies our theorem. We can repeat the same proof for points with negative first coordinate, as well as points with positive or negative second coordinate. Since this covers all points in $\mathbb{S}^1$, we have proven our theorem.
\end{proof}

This is good because it is going to allow us to take questions about the fundamental group of $\mathbb{S}^1$ and reduce them to questions about $\mathbb{R}$.

\begin{definition}
Let $f: X \rightarrow \mathbb{S}^1$ be a map. A \textbf{lift} of $f$ is a map $\overset{\sim}{f}: X \rightarrow \mathbb{R}$ such that 

$$ f = p \circ \overset{\sim}{f} $$
\end{definition}

We are now going to show that any \textit{loop} in $\mathbb{S}^1$ lifts to a \textit{path} in $\mathbb{R}$. It is important to see that it will not always lift to a loop, meaning sometimes our lift will have different starting and ending points. First, let us fix the point $b_0 = (1,0)$ as our base point for $\mathbb{S}^1$. Then we see that $p^{-1}(b_0)$ is the set $\mathbb{Z}$ of integers.

Now let us describe our plan of attack. First we will prove that all loops lift to paths, then we will prove that path homotopies lift to path homotopies, that is, two loops in $\mathbb{S}^1$ are path homotopic if and only if their lifts are. Since we have already shown that any two paths in $\mathbb{R}$ are path homotopic, this will imply that two loops in $\mathbb{S}^1$ are path homotopic if and only if their lifts end at the same point.

\begin{lemma}
Let $f: I \rightarrow \mathbb{S}^1$ be a loop based at $b_0$, then there exists a unique lift of $f$, $\overset{\sim}{f}: I \rightarrow \mathbb{R}$ such that $f(0) = 0$.
\end{lemma}
\begin{proof}
We will not use the fact that lifts are unique anywhere so we skip that part of the proof.

To start, we cover $\mathbb{S}^1$ with the open subsets from Theorem 20.1. Call this cover $\mathcal{U}$. By applying problem 5 from Homework 7 to the open cover

$$ f^{-1}(\mathcal{U}) = \{f^{-1}(U) : U \in \mathcal{U}\} $$

we obtain a number $\delta >0$ such that for any closed interval, $J \subseteq I$ of length less than $\delta$, $f(J)$ is contained in some element of $\mathcal{U}$. So we can pick $s_0 < s_1 < \dots < s_n$, such that $s_0 = 0$, $s_1 = 1$ and $f([s_i, s_{i+1}])$ is contained in an element of $\mathcal{U}$. This will allow us to build our lift step by step.

First, we set $\overset{\sim}{f}(0) = 0$. Now, assume we have defined for all $s \in [0, s_i]$, we will show how to define it for $[0,s_{i+1}]$. Let $U$ be an element of $\mathcal{U}$ such that $f([s_i, s_{i+1}]) \subseteq U$. We can write

$$ p^{-1}(U) = \dots \overset{\cdot}{\cup} V_{-1} \overset{\cdot}{\cup} V_0 \overset{\cdot}{\cup} V_1 \overset{\cdot}{\cup} \dots $$

Let $V_i$ the one that contains $\overset{\sim}{f}(s_i)$. Since $p$ is a homeomorphism when restricted to $V_i$, we can define

$$ \overset{\sim}{f}(s) = (p_{|V_i})^{-1}(f(s)) $$

for $s \in [s_i, s_{i+1}]$

Note that

$$ (p \circ \overset{\sim}{f})(s) = p \circ (p_{|V_i})^{-1}(f(s)) = f(s) $$

so $f = (p_{|V_i})^{-1}(f(s))$ for $s \in [s_i,s_{i+1}]$

Then we use the gluing lemma to define $\overset{\sim}{f}$ for $[0,s_{i+1}]$. We continue this until we have defined $\overset{\sim}{f}$ for all of $I$. The fact that $f = p \circ \overset{\sim}{f}$ follows since it held during each step.

\end{proof}

This and the following lemma are the main keys.

\begin{lemma}
Let $H: I \times I \rightarrow \mathbb{S}^1$ be a path homotopy between loops based at $b_0$. Then there is a unique lift of $H$, $\overset{\sim}{H}: I \times I \rightarrow \mathbb{R}$ such that $H(0,0) = 0$. The lift $\overset{\sim}{H}$ is a path homotopy as well.
\end{lemma}

Because of the similarities to the last lemma, we will not go through this one in detail. The main difference is rather than breaking up $I$ into closed intervals, we break up $I \times I$ into smaller closed squares. This makes things slightly trickier since we are no longer gluing at a single point during each step, but it ends up working out fine. For details, look at page 343 of Munkres. Even though the proof in the book is for a general covering space, you should be able to see how it applies to our case.

We are now finally in a position to state our main theorem.

\begin{theorem}
$\pi(\mathbb{S}^1,b_0)$ is isomorphic to $(\mathbb{Z},+)$.
\end{theorem}

\begin{proof}
For each $n \in \mathbb{Z}$, let $\gamma_n: I \rightarrow \mathbb{S}^1$ be the loop defined by

$$ \gamma_n(t) = (\cos(2 n \pi t), \sin ( 2 n \pi t)) $$

One can intuitively see (and prove by writing out explicit homotopies) that

$$ [\gamma_n] \ast [\gamma_m] = [\gamma_{n + m}] $$

Thus the map $\phi: (\mathbb{Z},+) \rightarrow \pi(\mathbb{S}^1,b_0)$ defined by 

$$ \phi(n) = \gamma_n $$

is a homomorphism. We will show that it is an isomorphism. To show this, we must show $\phi$ is bijective.

Showing that $\phi$ is surjective is the same as showing that every loop is path homotopic to $\gamma_n$ for some $n \in \mathbb{Z}$ and showing it is injective is the same as showing that $\gamma_n$ is not path homotopic to $\gamma_m$ for $n \neq m$.

First we show every loop is path homotopic to $\gamma_n$ for some $n$.

It is easy to see that the $\gamma_n$ lift to paths $\overset{\sim}{\gamma_n}: I \rightarrow \mathbb{R}$ defined by

$$ \overset{\sim}{\gamma_n}(t) = nt $$

Let $f$ be a loop based at $b_0$. By Lemma 20.3, $f$ lifts to a path $\overset{\sim}{f}$ in $\mathbb{R}$ starting at $0$. Since $f$ ends at $b_0$, $\overset{\sim}{f}$ must end at $n$ for some $n \in \mathbb{Z}$.

Since $\overset{\sim}{f}$ and $\overset{\sim}{\gamma_n}$ have the same initial and final points, $\overset{\sim}{f}$ is path homotopic to $\gamma_n$. By problem 1 on Homework 7, this implies that $p \circ \overset{\sim}{f}$ and $p \circ \overset{\sim}{\gamma_n}$ are path homotopic. But $p \circ \overset{\sim}{f} = f$ and $p \circ \overset{\sim}{\gamma_n} = \gamma_n$ so we have proven $f$ and $\gamma_n$ are path homotopic.

Now we show $\gamma_n$ is not path homotopic to $\gamma_m$ for $n \neq m$.

Suppose, for the sake of contradiction, that $H$ is a path homotopy from $\gamma_n$ to $\gamma_m$ with $n \neq m$. Then by Lemma 20.4, $H$ lifts to a path homotopy from $\overset{\sim}{\gamma_n}$ to $\overset{\sim}{\gamma_m}$. But $\overset{\sim}{\gamma_n}$ and $\overset{\sim}{\gamma_m}$ have different end points, a contradiction.
\end{proof}


\section{Applications of the Fundamental Group}

Today we will be giving applications of the fundamental group to prove three famous theorems in math. To make sure we have enough time to do all three, we will skip the proofs of a few of the lemmas. Don't think this means they are too advanced, you can certainly look up proofs and understand them, it is purely to save time.

\subsection{The Brouwer Fixed Point Theorem}

Our first application is called the Brouwer Fixed point theorem. First, let us prove a simple lemma. Recall the definition of the closed ball,

$$ \textbf{B}^2 = \{x \in \mathbb{R}^2 : |x| \leq 1\} $$

Then clearly $\mathbb{S}^1 \subseteq \textbf{B}^2$. We see that $\textbf{B}^2$ is contractible by the following homotopy,

$$ H(x,t) = tx $$

This is a homotopy from the constant path at the origin to the identity. Therefore $\textbf{B}^2$ is simply connected by Homework 7, Problem 1.

\begin{lemma}
There does not exist a retraction $r: \textbf{B}^2 \rightarrow \mathbb{S}^1$.
\end{lemma}

\begin{proof}
Suppose, for the sake of contradiction, that there exists a retraction $r: \textbf{B}^2 \rightarrow \mathbb{S}^1$. Let $b_0 = (1,0)$. By Homework 7, Problem 4, the homomorphism $r_*: \pi(\textbf{B}^2,b_0) \rightarrow \pi(\mathbb{S}^1,b_0)$ is surjective. But $\pi(\textbf{B}^2,b_0)$ is a single element and $\pi(\mathbb{S}^1,b_0)$ is infinite, a contradiction.
\end{proof}

\begin{theorem}[\textbf{Brouwer Fixed Point Theorem}]
Let $f: \textbf{B}^2 \rightarrow \textbf{B}^2$ be a continuous map. Then there exists an $x \in \textbf{B}^2$ such that $f(x) = x$. Such an $x$ is called a \textbf{fixed point} of $f$.
\end{theorem}

\begin{proof}
Suppose, for the sake of contradiction, that there exists a function $f: \textbf{B}^2 \rightarrow \textbf{B}^2$ with no fixed points. We will use $f$ to construct a retraction to $\mathbb{S}^1$, giving us our contradiction. We define $r: \textbf{B}^2 \rightarrow \mathbb{S}^1$ by the following procedure: for $x \in \textbf{B}^2$, draw the ray that starts at $x$ and passes through $f(x)$ and let $r(x)$ be the point where that ray meets $\mathbb{S}^1$.

That description is intuitive but it is maybe not so obvious that $r$ is continuous, so we write out $r$ explicitly here:

$$ r(x) = \left(\frac{1+|f(x)|}{|x| + |f(x)|} - 1\right)f(x) + \left(\frac{1+|f(x)|}{|x| + |f(x)|}\right) x $$

To see that $r$ is a retraction, you can draw out the intuitive description of $r$ or just plug into our explicit equation.

\end{proof}

This theorem is true if we replace $\textbf{B}^2$ with $\textbf{B}^n$ but it requires developing stronger algebraic topology tools.

\subsection{Fundamental Theorem of Algebra}

Our next application is the famous Fundamental Theorem of Algebra. Before we state it, let us do a lightning speed review of the complex numbers.

The complex numbers are defined as $\mathbb{C} = \{a + bi : a,b \in \mathbb{R}\}$ where $i^2 = -1$. We give a topology to $\mathbb{C}$ by identifying it with $\mathbb{R}^2$ in the natural way:

$$ (a,b) = a + bi $$

For the rest of these notes, we will think of $\mathbb{S}^1$ as a subspace of $\mathbb{C}$.

We recall the following famous formula,

\begin{theorem}[\textbf{Euler's Formula}]
$$ e^{ix} = \cos (x) + \sin (x) $$
\end{theorem}

So our loops from last time can be written,

$$ \gamma_n(t) = e^{2 \pi i n t} $$

Recall that we call a map \textbf{nullhomotopic} if it is homotopic to a constant map.

\begin{lemma}
If $f: \mathbb{S}^1 \rightarrow \mathbb{S}^1$ is nullhomotopic, then $f_*$ is the map that sends everything to the identity.
\end{lemma}

We are going to skip the proof of this one to save time. 

\begin{lemma}
The map $p_n: \mathbb{S}^1 \rightarrow \mathbb{S}^1$ defined by 

$$ p_n(z) = z^n $$

is not nullhomotopic. (remember that multiplication is defined since $\mathbb{S}^1 \subseteq \mathbb{C}$)
\end{lemma}

\begin{proof}
By the previous lemma, it suffices to show that $p_{n*}$ is not constant. We see that 

$$ p_n(\gamma_1(t)) = (e^{2 \pi i t})^n = e^{2 \pi i n t} = \gamma_n(t) $$

Thus, $p_{n*}([\gamma_1]) = [\gamma_n]$. So $p_n$ is not nullhomotopic.
\end{proof}

We will now use this to prove the fundamental theorem of algebra.

\begin{theorem}[Fundamental Theorem of Algebra]
Let $p(z) = z^n + a_{n-1} z^{n-1} + \dots + a_0$ be a polynomial with coefficients in $\mathbb{C}$ with $n \geq 1$. There exists a $z_0 \in \mathbb{C}$ such that $p(z_0) = 0$. Such a point $z_0$ is called a \textbf{zero} of $p$.
\end{theorem}
\begin{proof}
Suppose, for the sake of contradiction, that $p$ has no zeros. Let us define a map $f: \mathbb{S}^1 \rightarrow \mathbb{S}^1$ by

$$ f(z) = \frac{p(z)}{|p(z)|} $$

We proceed in two steps. First we will show that $f$ is homotopic to the $n^{th}$ power map $p_n$. Second, we will show that $f$ is nullhomotopic. By transitivity of homotopies, this will imply that $p_n$ is nullhomotopic, giving us our contradiction.

For $\epsilon \geq 0$ define $p_{\epsilon}: \Bbb{C} \mapsto \Bbb{C}$ by

$$p_\epsilon(z) = z^n + \epsilon a_{n-1}z^{n-1} + \epsilon^2 a_{n-2}z^{n-2} + \dots + \epsilon^{n-1}a_1 z + \epsilon^{n} a_0$$

We see that for $\epsilon > 0$, $p_\epsilon(z) = \epsilon^n p(\frac{z}{\epsilon})$ and $p_0(z) = z^n$. Therefore, for any $\epsilon \geq 0$, $p_\epsilon(z) \neq 0$ for all $z \in \Bbb{C} \setminus \{0\}$ since $p$ has no zeros and $p_0(z) = z^n$. So for any $\epsilon \geq 0$, we can define the following homotopy: $H: \mathbb{S}^1 \times I \rightarrow \mathbb{S}^1$ by

$$ H(z,t) = \frac{p_t(z)}{|p_t(z)|} $$


$H$ is a homotopy from $p_n$ to $f$.

Now we define another homotopy: $H': \mathbb{S}^1 \times I \rightarrow \mathbb{S}^1$ by

$$ H'(z,t) = \frac{p(tz)}{|p(tz)|} $$

$H'$ is a homotopy from the constant map sending everything to $\frac{p(0)}{|p(0)|}$ to $f$. Thus $f$ is nullhomotopic, completing our proof.
\end{proof}

\subsection{Borsuk-Ulam Theorem}
Let $f: \mathbb{S}^n \rightarrow \mathbb{S}^m$ be a map. We say $f$ is \textbf{antipode-preserving} if $f(-x) = -f(x)$.

\begin{lemma}
Let $f: \mathbb{S}^1 \rightarrow \mathbb{S}^1$ be continuous and antipode-preserving. Then $f$ is not nullhomotopic.
\end{lemma}
We will be skipping the proof of this lemma to save time.

\begin{lemma}
There is no continuous antipode-preserving map $f: \mathbb{S}^2 \rightarrow \mathbb{S}^1$.
\end{lemma}

\begin{proof}
Suppose for the sake of contradiction that $f: \mathbb{S}^2 \rightarrow \mathbb{S}^1$ is continuous and antipode-preserving. Then $f$ restricted to the equator of $\mathbb{S}^2$, $E$, gives an antipode preserving map from $E \cong \mathbb{S}^1$ to $\mathbb{S}^1$. The top half of the sphere, $Q$, is homeomorphic to $\textbf{B}^2$ and $\textbf{B}^2$ is contractible, so $p$ restricted to $Q$ is nullhomotopic. Let $H: Q \times I \rightarrow \mathbb{S}^1$ be a homotopy from $p_{|Q}$ to a constant map. Then $H_{E \times I}$ is a homotopy from $p_{|E}$ to a constant map. So $p_{|E}$ is nullhomotopic, a contradiction by the previous lemma.
\end{proof}

\begin{theorem}[Borsuk-Ulam Theorem]
Let $f: \mathbb{S}^2 \rightarrow \mathbb{R}^2$ be a continuous map. Then there exists an $x \in \mathbb{S}^2$ such that $f(x) = f(-x)$.
\end{theorem}

\begin{proof}
Suppose, for the sake of contradiction, that there does not exist such an $x$. Then the map

$$ g(x) = \frac{f(x)-f(-x)}{|f(x)-f(-x)|} $$

is a continuous antipode preserving map from $\mathbb{S}^2$ to $\mathbb{S}^1$.

\end{proof}


\end{document}