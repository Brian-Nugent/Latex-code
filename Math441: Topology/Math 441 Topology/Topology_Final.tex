%This is my super simple Real Analysis Homework template

\documentclass{article}

\usepackage[margin=1in]{geometry}

\usepackage[utf8]{inputenc}
\usepackage[english]{babel}
\usepackage{tikz-cd}

\usepackage[]{amsthm} %lets us use \begin{proof}
\usepackage[]{amssymb} %gives us the character \varnothing

\usepackage{tikz}

\usepackage{amsmath}

\newcommand{\tens}[1]{%
  \mathbin{\mathop{\otimes}\limits_{#1}}%
}

\usepackage[margin=1in]{geometry}


\usepackage{faktor}\usepackage{amsmath}\usepackage{amssymb}

\makeatletter
\DeclareRobustCommand*{\mfaktor}[3][]
{
   { \mathpalette{\mfaktor@impl@}{{#1}{#2}{#3}} }
}
\newcommand*{\mfaktor@impl@}[2]{\mfaktor@impl#1#2}
\newcommand*{\mfaktor@impl}[4]{
   \settoheight{\faktor@zaehlerhoehe}{\ensuremath{#1#2{#3}}}%
   \settoheight{\faktor@nennerhoehe}{\ensuremath{#1#2{#4}}}%
      \raisebox{-0.5\faktor@zaehlerhoehe}{\ensuremath{#1#2{#3}}}%
      \mkern-4mu\diagdown\mkern-5mu%
      \raisebox{0.5\faktor@nennerhoehe}{\ensuremath{#1#2{#4}}}%
}
\makeatother

\newtheorem*{lemma}{Lemma}
\newtheorem*{theorem}{Theorem}

\setlength\parindent{0pt}


\newcommand{\p}{\mathfrak{p}}
\newcommand{\m}{\mathfrak{m}}

\newcommand{\Ker}{\textrm{Ker}}
\newcommand{\Coker}{\textrm{Coker}}
\newcommand{\coker}{\textrm{coker}}
\newcommand{\Imm}{\textrm{Im}}
\newcommand{\im}{\textrm{im}}
\newcommand{\Coim}{\textrm{Coim}}
\newcommand{\coim}{\textrm{coim}}
\newcommand{\jump}{\vspace{0.5 in}}
\newcommand{\bigjump}{\vspace{1.7 in}}

\DeclareMathOperator{\gal}{Gal}

%This information doesn't actually show up on your document unless you use the maketitle command below

\begin{document}

\begin{center}
   {\huge Topology Final}
\end{center}

\vspace{0.3 in}

Name: \underline{\hspace{6cm}}

\vspace{0.2cm}

\subsection*{Problem 1 [20 points]}
(a) Give the definition of a topological space.

\bigjump

(b) Give the definition of connected.

\bigjump

(c) Give the definition of compact.

\bigjump

(d) Give the definition of simply connected.

\newpage

\subsection*{Problem 2 [20 points]}
For this problem, just write true or false. \textbf{You do NOT need to justify your answers.}

\vspace{0.5cm}

(a) True or False: If $X$ is compact and $q: X \rightarrow Y$ is a quotient map, then $Y$ is compact.

\bigjump

(b) True or False: Let $A$ be a connected subspace of $X$. Then the closure of $A$ is connected.

\bigjump

(c) True or False: Let $A$ be a subspace of $X$. If the closure of $A$ is connected, then $A$ is connected.

\bigjump

(d) True or False: The symbols 0 and $\infty$ (subspaces of $\mathbb{R}^2$) are homeomorphic.

\newpage


\subsection*{Problem 3 [20 points]} This question is asking you to prove statements that we have proven in class, do not just cite them.

\vspace{1cm}

(a) Let $f: X \rightarrow Y$ be a continuous function. Prove that if $X$ is connected, then $f(X)$ is connected.

\vspace{10cm}

(b) Let $Z$ be a compact subspace of a Hausdorff space $X$. Prove that $Z$ is closed.


\newpage
\subsection*{Problem 4 [40 points]}

\textbf{Pick 2 of the following problems to solve.} If you submit more than two, mark which two you would like graded.

\vspace{1cm}

(a) Let $q: X \rightarrow \mathbb{S}^1$ be a continuous map where $X$ is simply connected and let $i: \mathbb{S}^1 \rightarrow \mathbb{S}^1$ be the identity map. Prove that there does not exist a continuous map $\overset{\sim}{i}: \mathbb{S}^1 \rightarrow X$ such that $q \circ \overset{\sim}{i} = i$. 

\vspace{1cm}

(b) Let $X$ be a topological space and suppose there is a subspace $Z \subseteq X$ that is infinite, closed and has the discrete topology (in the subspace topology). Prove that $X$ is not compact.

\vspace{1cm}

(c) Let $A$ and $B$ be compact subspaces of a Hausdorff space $X$. Prove that $A \cup B$ and $A \cap B$ are compact.

\end{document}