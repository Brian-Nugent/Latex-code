%This is my super simple Real Analysis Homework template

\documentclass{article}

\usepackage[margin=1in]{geometry}

\usepackage[utf8]{inputenc}
\usepackage[english]{babel}
\usepackage{tikz-cd}

\usepackage[]{amsthm} %lets us use \begin{proof}
\usepackage[]{amssymb} %gives us the character \varnothing

\usepackage{tikz}

\usepackage{amsmath}

\newcommand{\tens}[1]{%
  \mathbin{\mathop{\otimes}\limits_{#1}}%
}

\usepackage[margin=1in]{geometry}


\usepackage{faktor}\usepackage{amsmath}\usepackage{amssymb}

\makeatletter
\DeclareRobustCommand*{\mfaktor}[3][]
{
   { \mathpalette{\mfaktor@impl@}{{#1}{#2}{#3}} }
}
\newcommand*{\mfaktor@impl@}[2]{\mfaktor@impl#1#2}
\newcommand*{\mfaktor@impl}[4]{
   \settoheight{\faktor@zaehlerhoehe}{\ensuremath{#1#2{#3}}}%
   \settoheight{\faktor@nennerhoehe}{\ensuremath{#1#2{#4}}}%
      \raisebox{-0.5\faktor@zaehlerhoehe}{\ensuremath{#1#2{#3}}}%
      \mkern-4mu\diagdown\mkern-5mu%
      \raisebox{0.5\faktor@nennerhoehe}{\ensuremath{#1#2{#4}}}%
}
\makeatother

\newtheorem*{lemma}{Lemma}
\newtheorem*{theorem}{Theorem}

\setlength\parindent{0pt}


\newcommand{\p}{\mathfrak{p}}
\newcommand{\m}{\mathfrak{m}}

\newcommand{\Ker}{\textrm{Ker}}
\newcommand{\Coker}{\textrm{Coker}}
\newcommand{\coker}{\textrm{coker}}
\newcommand{\Imm}{\textrm{Im}}
\newcommand{\im}{\textrm{im}}
\newcommand{\Coim}{\textrm{Coim}}
\newcommand{\coim}{\textrm{coim}}
\newcommand{\jump}{\vspace{0.5 in}}
\newcommand{\bigjump}{\vspace{1.7 in}}

\DeclareMathOperator{\gal}{Gal}

%This information doesn't actually show up on your document unless you use the maketitle command below

\begin{document}

\begin{center}
   {\huge Topology Midterm}
\end{center}

\vspace{0.3 in}

Name: \underline{\hspace{6cm}}

\vspace{0.2cm}


\subsection*{Problem 1 [20 points]}
For this problem, just write true or false. \textbf{You do NOT need to justify your answers.}

\vspace{0.5cm}

a) True or False: In a topological space, arbitrary intersections of closed sets are closed.

\bigjump

b) True or False: If $\mathcal{T}_1$ and $\mathcal{T}_2$ are topologies on a set $X$, then $\mathcal{T}_1 \cup \mathcal{T}_2$ is a topology on $X$.

\bigjump

c) True or False: If $\mathcal{T}_1$ and $\mathcal{T}_2$ are topologies on a set $X$, then $\mathcal{T}_1 \cap \mathcal{T}_2$ is a topology on $X$.

\bigjump

d) True or False: If $Y$ is a subspace of $X$, then a subset $U \subseteq Y$ is open in $Y$ if and only if $U$ is open in $X$.

\newpage

\subsection*{Problem 2 [20 points]}

A map $f: X \rightarrow Y$ is called \textbf{open} if the image on any open subset of $X$ is an open subset of $Y$.

\vspace{0.5cm}

a) Prove that an open, surjective, continuous map is a quotient map.

\vspace{8cm}

b) Let $\mathbb{R}_{cof}$ be $\mathbb{R}$ with the cofinite topology. Consider the function $f: \mathbb{R}_{cof} \rightarrow \mathbb{R}_{cof}$ defined by

$$ f(x) = \sin (x) $$

Is $f$ continuous? Is it open? Prove your answer.

\jump

\newpage

\subsection*{Problem 3 [20 points]} Consider the set of real numbers $\mathbb{R}$ with the following topology:

$$ \mathcal{T} = \{ U \subseteq \mathbb{R} : \forall x \in U, -x \in U \} $$

You do not need to prove that $\mathcal{T}$ is a topology.

\jump

a) Let $A = [-1,3)$. Is $A$ open? Is $A$ closed? Prove your answer.

\bigjump

b) Determine the interior and closure of $A$. Prove your answer.

\bigjump

c) What are the limit points of $A$. \textbf{You do NOT need to prove your answer.}

\bigjump

d) Is $A$ Hausdorff with the subspace topology? Prove your answer.


\newpage
\subsection*{Problem 4 [20 points]}
Let $X$ be a topological space, we define a subset $D \subseteq X \times X$ by 

$$ D = \{(x,x) : x \in X \} $$

\vspace{0.15 cm}

Prove that $X$ is Hausdorff if and only if $D$ is closed in $X \times X$.

\vspace{15 cm}

\subsection*{Bonus Problem [2 points]}

What is one thing you like and one thing you would like to see changed about the class?

\end{document}